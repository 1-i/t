\section{Momento angolare}

\subsection{Noncommutativit\`a dei generatori di rotazione}

Consideriamo un vettore tridimensionale $A$, ed un secondo vettore, $A'$, ottenuto mediante una rotazione di $A$ di un angolo $\varphi$ attorno all'asse $z$. Se chiamiamo $\vartheta$ l'angolo tra la proiezione di $A$ sul piano $xy$ e l'asse $y$, possiamo scrivere le componenti di $A'$ in funzione di $\varphi, \vartheta$ e delle componenti di $A$.

Innanzitutto, dal momento che la rotazione avviene attorno all'asse $z$, \`e evidente che:

	\begin{equation} \label{eq:rotMatrixZ}
		A'_z = A_z
	\end{equation}

Dal momento che $A'$ deve essere uguale in modulo ad $A$ e vale la \eqref{eq:rotMatrixZ}, anche i moduli delle proiezioni di $A$, $A'$ sul piano $xy$ devono essere uguali. Possiamo quindi calcolare $A'_x$ come:

	\begin{equation}
		A'_x = \left | A \right |_{xy} \cos \alpha
	\end{equation}

Dove $\alpha$ \`e l'angolo tra l'asse $x$ ed $A'$ proiettato sul piano $xy$. Esplicitamente:

	\begin{equation}
		A'_x = \cos(\varphi + \vartheta) \sqrt{A_x^2 + A_y^2} = (\cos \varphi \cos \vartheta - \sin \varphi \sin \vartheta)
			\sqrt{A_x^2 + A_y^2}
	\end{equation}

Da cui:

	\begin{equation} \label{eq:rotMatrixX}
		A'_x = A_x \cos \varphi - A_y \sin \varphi
	\end{equation}

In maniera analoga si calcola:

	\begin{equation}
		A'_y = \sin( \vartheta + \varphi ) \sqrt{A_x^2 + A_y^2} = (\sin \varphi \cos \vartheta + \sin \vartheta \cos \varphi)
			\sqrt{A_x^2 + A_y^2}
	\end{equation}

Ovvero:

	\begin{equation} \label{eq:rotMatrixY}
		A'_y = A_x \sin \varphi + A_y \cos \varphi
	\end{equation}

Mettendo a sistema la \eqref{eq:rotMatrixZ}, la \eqref{eq:rotMatrixX}, e la \eqref{eq:rotMatrixY} otteniamo:

	\begin{equation}
		\left ( \begin{array}{c}
			A'_x \\ A'_y \\ A'_z
		\end{array} \right ) =
		\left ( \begin{array}{c c c}
			\cos \varphi & -\sin \varphi & 0 \\
			\sin \varphi & \cos \varphi & 0 \\
			0 & 0 & 1 \\
		\end{array} \right )
		\left ( \begin{array}{c}
			A_x \\ A_y \\ A_z
		\end{array} \right )
	\end{equation}

La matrice che ruota il vettore di un angolo $\varphi$ attorno all'asse $z$ \`e quindi data da:

	\begin{equation}
		S(\varphi k) = 
		\left ( \begin{array}{c c c}
			\cos \varphi & -\sin \varphi & 0 \\
			\sin \varphi & \cos \varphi & 0 \\
			0 & 0 & 1 \\
		\end{array} \right )
	\end{equation}

Sostituendo in maniera ciclica $x, y, z$, otteniamo anche espressioni per le matrici di rotazione intorno agli assi $x$ ed $y$:
	
	\begin{equation}
		S(\varphi i) = 
		\left ( \begin{array}{c c c}
			1 & 0 & 0 \\
			0 & \cos \varphi & -\sin \varphi \\
			0 & \sin \varphi & \cos \varphi \\
		\end{array} \right )
	\end{equation}

	\begin{equation}
		S(\varphi j) = 
		\left ( \begin{array}{c c c}
			\cos \varphi & 0 & \sin \varphi \\
			0 & 1 & 0 \\
			-\sin \varphi & 0 & \cos \varphi \\
		\end{array} \right )
	\end{equation}

Per verificare la non-commutativit\`a delle rotazioni, prendiamo lo sviluppo di Taylor al secondo ordine delle precedenti per un piccolo angolo $\Delta \varphi$:

	\begin{equation}
		S(\Delta \varphi i) = 
		\left ( \begin{array}{c c c}
			1 & 0 & 0 \\
			0 & 1 - \Delta \varphi^2 / 2 & - \Delta \varphi \\
			0 & \Delta \varphi & 1 - \Delta \varphi^2 / 2 \\
		\end{array} \right )
	\end{equation}

	\begin{equation}
		S(\Delta \varphi j) = 
		\left ( \begin{array}{c c c}
			1 - \Delta \varphi^2 / 2 & 0 & \Delta \varphi \\
			0 & 1 & 0 \\
			-\Delta \varphi & 0 & 1 - \Delta \varphi^2 / 2 \\
		\end{array} \right )
	\end{equation}

E calcoliamo:

	\begin{equation}
		S(\Delta \varphi i) S(\Delta \varphi j) - S(\Delta \varphi j) S(\Delta \varphi i) =
			\left ( \begin{array}{c c c}
				0 & - \Delta \varphi^2 & 0 \\
				\Delta \varphi^2 & 0 & 0 \\
				0 & 0 & 0 \\
			\end{array} \right )
	\end{equation}

da cui deduciamo che gli operatori di rotazione non commutano. Se eliminiamo tutti i termini oltre il secondo ordine, allora:
	
	\begin{equation}
		S(\Delta \varphi^2 k) = 
		\left ( \begin{array}{c c c}
			1 & - \Delta \varphi^2 & 0\\
			\Delta \varphi^2 & 1 & 0 \\
			0 & 0 & 1 \\
		\end{array} \right )
	\end{equation}

possiamo allora scrivere anche:

	\begin{equation} \label{eq:rotationOperatorsComm}
		S(\Delta \varphi i) S(\Delta \varphi j) - S(\Delta \varphi j) S(\Delta \varphi i) = S(\Delta \varphi^2 k) - I
	\end{equation}

Quest'ultima relazione \`e significativa per dedurre il commutatore dei generatori di rotazione. Infatti, per la definizione di generatore di rotazioni abbiamo:

	\begin{equation}
		\hat{R} ( \varphi i ) = e^{-i \hat{J}_x \varphi / \hbar } \quad \quad
		\quad \hat{R} ( \varphi j ) = e^{-i \hat{J}_y \varphi / \hbar } 
	\end{equation}

Dunque, prendendo lo sviluppo al secondo ordine degli operatori e sostituendoli nella \eqref{eq:rotationOperatorsComm} abbiamo:

	\begin{equation}
		\begin{array}{r l}
		& \left ( 1 - \frac{i \hat{J}_x \Delta \varphi}{\hbar} - \frac{1}{2} \left ( \frac{\hat{J}_x \Delta \varphi}{\hbar} \right ) ^ 2 \right ) \left ( 1 - \frac{i \hat{J}_y \Delta \varphi}{\hbar} - \frac{1}{2} \left ( \frac{\hat{J}_y \Delta \varphi}{\hbar} \right ) ^ 2 \right ) \\
		- & \left ( 1 - \frac{i \hat{J}_y \Delta \varphi}{\hbar} - \frac{1}{2} \left ( \frac{\hat{J}_y \Delta \varphi}{\hbar} \right ) ^ 2 \right ) \left ( 1 - \frac{i \hat{J}_x \Delta \varphi}{\hbar} - \frac{1}{2} \left ( \frac{\hat{J}_x \Delta \varphi}{\hbar} \right ) ^ 2 \right ) \\ 
		= & \left ( 1 - \frac{i \hat{J}_z \Delta \varphi^2}{\hbar} \right ) - 1
		\end{array}
	\end{equation}

Uguagliando gli unici termini al secondo ordine che rimangono si ottiene:

	\begin{equation}
		\hat{J}_x \hat{J}_y - \hat{J}_y \hat{J}_x = i \hbar \hat{J}_z
	\end{equation}

Ovvero:

	\begin{equation}
		\left [ \hat{J}_x, \hat{J}_y \right ] = i \hbar \hat{J}_z
	\end{equation}

Naturalmente, valgono anche le analoghe alla precedente ottenute per rotazione ciclica degli indici:

	\begin{equation}
		\left [ \hat{J}_y, \hat{J}_z \right ] = i \hbar \hat{J}_x
	\end{equation}

	\begin{equation}
		\left [ \hat{J}_z, \hat{J}_x \right ] = i \hbar \hat{J}_y
	\end{equation}

\subsection{Operatori commutativi}

Consideriamo due operatori lineari $\hat{A}$, $\hat{B}$ che soddisfino

	\begin{equation} \label{eq:commutativeOperators}
		\left [ \hat{A}, \hat{B} \right ] \overset{def}{=} \hat{A}\hat{B} - \hat{B}\hat{A} = 0
	\end{equation}

Supponiamo che esista un unico stato $\ket{a}$ che sia autovettore di $\hat{A}$ con autovalore $a$:

	\begin{equation} \label{eq:assumeUniqEigenstate}
		\hat{A}\ket{a} = a \ket{a}
	\end{equation}

Applicando $\hat{B}$ ad entrambi i lati della \eqref{eq:assumeUniqEigenstate} otteniamo

	\begin{equation}
		\hat{B}\hat{A}\ket{a} = \hat{B}a\ket{a}
	\end{equation}

Per la \eqref{eq:commutativeOperators} e per linearit\`a di $\hat{B}$ possiamo scrivere:

	\begin{equation}
		\hat{A}\hat{B} \ket{a} = a \hat{B} \ket{a}
	\end{equation}

che interpretiamo come:

	\begin{equation} \label{eq:commonEigenstate}
		\hat{A} \left ( \hat{B} \ket{a} \right ) = a \left ( \hat{B} \ket{a} \right )
	\end{equation}

La \eqref{eq:commonEigenstate} ci dice che lo stato $\hat{B} \ket{a}$ \`e un autostato dell'operatore $\hat{A}$ con autovalore $a$. Ma avevamo supposto che ci fosse uno solo di questi stati, concludiamo quindi che deve necessariamente essere:

	\begin{equation} \label{eq:sharedEigenstate}
		\hat{B}\ket{a} = b \ket{a}
	\end{equation}

dove $b$ \`e una costante, dal momento che se $\ket{a}$ soddisfa la \eqref{eq:assumeUniqEigenstate}, anche $b \ket{a}$ per ogni $b$ la soddisfa analogamente. La \eqref{eq:sharedEigenstate} ci dice per\`o che $\ket{a}$ \`e anche un autostato di $\hat{B}$, con autovalore $b$. Per denotare uno stato di questo tipo, utilizzeremo la notazione $\ket{a, b}$ e diremo che $\hat{A}$, $\hat{B}$ hanno l'autostato $\ket{a, b}$ in comune. \\

Se esiste pi\`u di un autostato dell'operatore $\hat{A}$ con autovalore $a$, ci troviamo in un caso \textit{degenere}. In un caso \textit{degenere}, si possono sempre trovare combinazioni lineari degli stati degeneri di $\hat{A}$ che sono autostati dell'operatore $\hat{B}$, implicando che due operatori lineari autoaggiunti che commutano hanno sempre un insieme completo di autostati in comune. Questo risultato \`e una conseguenza del teorema spettrale.

Come abbiamo visto, i generatori di rotazioni attorno ad assi diversi non commutano, ma l'operatore:

	\begin{equation}
		\mathbf{\hat{J}^2} = \hat{J} \cdot \hat{J} = \hat{J}_x^2 + \hat{J}_y^2 + \hat{J}_z^2
	\end{equation}

commuta con ciascuno dei generatori. Per dimostrarlo, prendiamo ad esempio il generatore $\hat{J}_z$. Utilizzeremo l'identit\`a

	\begin{equation} \label{eq:commutatorsTrick}
		\left [ \hat{A}, \hat{B}\hat{C} \right ] = \hat{B} \left [ \hat{A}, \hat{C} \right ] + \left [ \hat{A}, \hat{B} \right ] \hat{C}
	\end{equation}

Scriviamo:

	\begin{equation}
		\left [ \hat{J}_z, \hat{J}_x^2 + \hat{J}_y^2 + \hat{J}_z^2 \right ] = \left [ \hat{J}_z, \hat{J}_x^2 \right ] + \left [ \hat{J}_z, \hat{J}_y^2 \right ]
	\end{equation}

Per la \eqref{eq:commutatorsTrick}, $\left [ \hat{J}_z, \hat{J}_x^2 + \hat{J}_y^2 + \hat{J}_z^2 \right ]$ si scrive come:

	\begin{equation}
	 \hat{J}_x \left [ \hat{J}_z, \hat{J}_x \right ] + \left [ \hat{J}_z , \hat{J}_x \right ] \hat{J}_x + \hat{J}_y \left [ \hat{J}_z, \hat{J}_y \right ] + \left [ \hat{J}_z , \hat{J}_y \right ] \hat{J}_y 
	\end{equation}

che diventa:

	\begin{equation}
		i \hbar \left ( \hat{J}_x \hat{J}_y + \hat{J}_y \hat{J}_x - \hat{J}_y \hat{J}_x - \hat{J}_x \hat{J}_y \right ) = 0
	\end{equation}

Ora, dal momento che l'operatore $\mathbf{\hat{J}^2}$ commuta con $\hat{J}_z$, devono avere autostati in comune. Siano i ket $\ket{\lambda}$, $\ket{m}$ tali che;

	\begin{equation} \label{eq:commutationRelation}
		\begin{array} {c}
			\mathbf{\hat{J}^2} \ket{\lambda, m} = \lambda \hbar^2 \ket{\lambda, m} \\
			\hat{J}_z \ket{\lambda, m} = m \hbar \ket{\lambda, m}
		\end{array}
	\end{equation}

Dove le dimensioni sono esplicitamente indicate, in maniera tale che $\lambda$ ed $m$ risultino adimensionali. Quindi, $\ket{\lambda, m}$ \`e uno stato per cui la misurazione della componente $z$ del momento angolare restituisce il valore $m \hbar$, mentre il modulo quadro del momento angolare risulta essere $\lambda \hbar^2$. Possiamo anche vedere che $\lambda \ge 0$, consistentemente con l'interpretazione per cui $\lambda \hbar^2$ sia il modulo quadro del momento angolare. Applichiamo $\bra{\lambda, m}$ ad entrambi i lati della \eqref{eq:commutationRelation}:

	\begin{equation}
		\mel{\lambda, m}{\mathbf{\hat{J}^2}}{\lambda, m} = \lambda \hbar^2 \braket{\lambda, m}{\lambda, m}
	\end{equation}

Come per ogni stato, $\braket{\lambda, m}{\lambda, m} = 1$. Scomponendo il membro sinistro, otterremo termini analoghi a:

	\begin{equation} 
		\mel{\lambda, m}{\hat{J}_x^2}{\lambda, m}
	\end{equation}

che possiamo vedere come:

	\begin{equation} \label{eq:genericMatrixElement}
		\left ( \bra{\lambda, m} \hat{J}_x \right ) \left ( \hat{J}_x \ket{\lambda, m} \right )
	\end{equation}

Non possiamo essere certi che esista un \textit{ket} normalizzato $\ket{\psi}$ tale che:

	\begin{equation}
		\hat{J}_x \ket{\lambda, m} = \ket{\psi}
	\end{equation}

ma certamente si pu\`o scrivere:

	\begin{equation}
		\hat{J}_x \ket{\lambda, m} = c \ket{\varphi}
	\end{equation}

Dal momento che vale la precedente, come dimostrato nella \eqref{eq:proofHard3}:

	\begin{equation}
		\bra{\lambda, m} \hat{J}_x = c^* \bra{\varphi}
	\end{equation}

Combinando le due precedenti la \eqref{eq:genericMatrixElement} diventa:

	\begin{equation} \label{eq:angMomGeZero}
		\left ( c^* \bra{\varphi} \right ) \left ( c \ket{\varphi} \right ) = c^*c \braket{\varphi}{\varphi} \ge 0
	\end{equation}

La disuguaglianza mostrata nella \eqref{eq:angMomGeZero} vale chiaramente anche per gli altri generatori di rotazione. Questo conclude il nostro argomento e dimostra che deve essere necessariamente $\lambda \ge 0$.

\subsection{Operatori di scala}

Introduciamo due operatori:

	\[
		\hat{J}_+ \quad e \quad \hat{J}_-
	\]

dove $\hat{J}_+$ \`e definito come:

	\[
		\hat{J}_+ = \hat{J}_x + i \hat{J}_y
	\]

Notiamo che non sono autoaggiunti, infatti:

	\[
		\hat{J}^\dagger_+ = \hat{J}^\dagger_x - i \hat{J}^\dagger_y = \hat{J}_-
	\]

L'utilit\`a degli operatori deriva dalle loro relazioni di commutazione con $\hat{J}_z$, infatti:

	\begin{equation}
		\left [ \hat{J}_z, \hat{J}_\pm \right ] = \left [ \hat{J}_z, \hat{J}_x \pm i \hat{J}_y \right ] = i \hbar \hat{J}_y \pm
			i ( -i \hbar \hat{J}_x ) = \pm \hbar \hat{J}_\pm
	\end{equation}

Utilizzando la precedente, possiamo calcolare:

	\begin{equation}
		\hat{J}_z \hat{J}_+ \ket{ \lambda, m} = ( \hat{J}_+ \hat{J}_z + \hbar \hat{J}_+ ) \ket{\lambda, m}
	\end{equation}

che diventa:

	\begin{equation}
		\hat{J}_z \hat{J}_+ \ket{\lambda, m} = (\hat{J}_+ m \hbar + \hbar \hat{J}_+ ) \ket{\lambda, m}
	\end{equation}

raccogliendo:

	\begin{equation}
		\hat{J}_z \left ( \hat{J}_+ \ket{\lambda, m} \right ) = (m + 1) \hbar \left ( \hat{J}_+ \ket{\lambda, m} \right )
	\end{equation}

Questo mostra che $\hat{J}_+ \ket{\lambda, m}$ \`e un autostato di $\hat{J}_z$ con autovalore $(m+1) \hbar$

Con analoghi calcoli possiamo far vedere che:

	\begin{equation}
		\hat{J}_z \left ( \hat{J}_- \ket{\lambda, m} \right ) = (m - 1) \hbar \left ( \hat{J}_- \ket{\lambda, m} \right )
	\end{equation}

Ovvero che $\hat{J}_- \ket{\lambda, m}$ \`e un autostato di $\hat{J}_z$ con autovalore $(m-1) \hbar$. Intuitivamente, i due operatori ``scalano'' lo stato su cui operano di un livello, e sono pertanto detti ``di scala''.
Notiamo che, come lecito aspettarsi, $\hat{J}_\pm \ket{\lambda, m}$ sono ancora autostati dell'operatore $\mathbf{\hat{J}^2}$ con autovalore $\lambda \hbar^2$. Infatti, siccome

	\begin{equation}
		\left [ \mathbf{\hat{J}^2}, J_x \right ] = 0 \quad e \quad \left [ \mathbf{\hat{J}^2}, J_y \right ] = 0
	\end{equation}

Gli operatori $\hat{J}_+, \hat{J}_-$ comutano con $\mathbf{\hat{J}^2}$, quindi:

	\begin{equation}
		\mathbf{\hat{J}^2} \left ( \hat{J}_\pm \ket{\lambda, m} \right ) = \hat{J}_\pm \mathbf{\hat{J}^2} \ket{\lambda, m} = \lambda \hbar^2 \left (
			\hat{J}_\pm \ket{\lambda, m} \right )
	\end{equation}

\subsection{Spettro degli autovalori}
Possiamo ora determinare gli autovalori $\lambda$ ed $m$. Un vincolo di natura fisica ci \`e dato dal fatto che il quadrato della proiezione del momento angolare su ciascuno degli assi non deve eccedere il modulo di $\mathbf{\hat{J}^2}$. Dal momento che:

	\begin{equation}
		\mel{\lambda, m}{\left ( \hat{J}^2_x + \hat{J}^2_y \right )}{\lambda, m} \ge 0
	\end{equation}

abbiamo:

	\begin{equation}
		\mel{\lambda, m}{\left ( \mathbf{\hat{J}^2} - \hat{J}^2_z \right )}{\lambda, m} = \left ( \lambda - m^2 \right ) \hbar^2 \braket{\lambda, m}{\lambda, m} \ge 0
	\end{equation}

che implica:

	\begin{equation}
		m^2 \le \lambda
	\end{equation}

Che \`e consistente con l'intuizione fisica concordando con la \eqref{eq:commutationRelation}. Sia ora $j$ il massimo possibile valore per $m$. Dobbiamo avere:

	\begin{equation}
		\hat{J}_+ \ket{\lambda, j} = 0
	\end{equation}

La precedente \`e necessaria perch\'e altrimenti $\hat{J}_+$ creerebbe uno stato $\ket{\lambda, j+1}$, violando la nostra assunzione per cui $j$ \`e il massimo autovalore per $\hat{J}_z$. Calcoliamo:

	\begin{equation}
		\hat{J}_- \hat{J}_+ = \left ( \hat{J}_x - i \hat{J}_y \right ) \left ( \hat{J}_x + i \hat{J}_y \right )
	\end{equation}

da cui:

	\begin{equation}
		\hat{J}_- \hat{J}_+ = \hat{J}^2_x + \hat{J}^2_y + i \left [ \hat{J}_x, \hat{J}_y \right ]
	\end{equation}

svolgendo il commutatore:

	\begin{equation} \label{eq:raisLowComp}
		\hat{J}_- \hat{J}_+	= \mathbf{\hat{J}^2} - \hat{J}_z^2 - \hbar \hat{J}_z
	\end{equation}

Possiamo vedere, applicando la \eqref{eq:raisLowComp}, che:

	\begin{equation}
		\hat{J}_- \hat{J}_+ \ket{\lambda, j} = \left ( \mathbf{\hat{J}^2} - \hat{J}_z^2 - \hbar \hat{J}_z \right ) \ket{\lambda, j}
	\end{equation}

che si riscrive come:

	\begin{equation}
		\hat{J}_- \hat{J}_+ = \left( \lambda - j^2 - j \right ) \hbar^2 \ket{\lambda, j} = 0
	\end{equation}

La precedente implica che:

	\begin{equation} \label{eq:lambdaVinc1}
		\lambda = j ( j + 1 )
	\end{equation}

Analogamente, se definiamo $j'$ come il minimo valore possibile per $m$ dovremo avere:

	\begin{equation}
		\hat{J}_- \ket{\lambda, j} = 0
	\end{equation}

Notiamo che:

	\begin{equation}
		\begin{array}{c c l}
			\hat{J}_+ \hat{J}_- & = & \left ( \hat{J}_x + i \hat{J}_y \right ) \left ( \hat{J}_x - i \hat{J}_y \right ) \\
			& = & \hat{J}^2_x + \hat{J}^2_y - i \left [ \hat{J}_x, \hat{J}_y \right ] \\
			& = & \mathbf{\hat{J}^2} - \hat{J}^2_z + \hbar \hat{J}_z
		\end{array}
	\end{equation}

Da cui possiamo dedurre:

	\begin{equation}
		\begin{array}{c c l}
			\hat{J}_+ \hat{J}_- \ket{\lambda, j'} & = & \left ( \mathbf{\hat{J}^2} - \hat{J}_z^2 + \hbar \hat{J}_z \right ) \ket{\lambda, j'}\\

		& = & \left ( \lambda - j'^2 + j' \right ) \hbar^2 \ket{\lambda, j'} = 0
		\end{array}
	\end{equation}

Quindi, deve essere:

	\begin{equation} \label{eq:lambdaVinc2}
		\lambda  = j'^2 - j'
	\end{equation}

Mettendo insieme la \eqref{eq:lambdaVinc1} e la \eqref{eq:lambdaVinc2} si ottiene:

	\begin{equation}
		j'^2 - j = j^2 + j
	\end {equation}

Gli unici valori che soddisfano la precedente sono:

	\[
		j' = j + 1 \quad \vee \quad j' = -j
	\]

Ma la prima contraddirrebbe la nostra assunzione per cui $j$ \`e il massimo valore possibile per $m$. Dunque il minimo valore possible per $m$ \`e $-j$. Consideriamo ora lo stato:

	\[
		\ket{\lambda, j}
	\]

Se applichiamo l'operatore $\hat{J}_-$ un numero sufficiente di volte, dobbiamo necessariamente raggiungere lo stato

	\[
		\ket{\lambda, -j}
	\]

Se cos\`i non fosse, dovremmo raggiungere uno stato diverso da $-j$, poniamo $h$, per cui valga $\hat{J}_- \ket{\lambda, h} = 0$, che \`e assurdo, perch\`e come abbiamo mostrato l'unico valore possible per $j'$ \`e $-j$. Dal momento che dobbiamo poter passare da $j$ a $-j$ in un numero finito ed intero di passi, dobbiamo avere:

	\begin{equation}
		\exists k \in \mathbb{N} : \quad j - (- j) = 2j = k
	\end{equation}

In altre parole, $j = k/2$ per qualche $k$ intero. I valori possibili per $j$ sono dunque limitati e dati da:

	\begin{equation}
		j \in \left \{ 0, \tfrac{1}{2}, 1, \tfrac{3}{2}, 2, \tfrac{5}{2}, ... \right \}
	\end{equation}

mentre i valori possibili di $m$ sono dati da:

	\begin{equation}
		m \in \left \{ j, j-1, j-2, ..., -j \right \}
	\end{equation}

Dal momento che i generatori di rotazione corrispondono agli operatori di spin, li indicheremo con $S_x, S_y, S_z$. Nell'utilizzare questa notazione, convenzionalmente si indicano gli stati come:

	\[
		\ket{s, m}
	\]

invece che come

	\[
		\ket{\lambda, m}
	\]

Prendendo $s$ tale che $s(s+1) = \lambda$ in maniera da avere:

	\begin{equation}
		\mathbf{\hat{S}^2} \ket{s, m} = s(s+1) \hbar^2 \ket{s, m}
	\end{equation}

In questo modo i valori di $s$ corrispondono allo spin totale della particella, ad esempio, i due stati

	\[
		\ket{+z} \quad \ket{-z}
	\]

che abbiamo utilizzato come base nella rappresentazione dei \textit{ket} si scrivono in questa notazione come:

	\[
		\ket{\tfrac{1}{2}, \tfrac{1}{2}} \quad \ket{\tfrac{1}{2}, - \tfrac{1}{2}}
	\]
