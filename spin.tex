\section{Particelle di spin $\frac{1}{2}$}

Avendo stabilito un formalismo matematico che ci consenta di operare sui fenomeni fisici, passiamo a considerare alcuni casi semplici.
Nel 1922, Otto Stern e Walther Gerlach effettuarono un esperimento che fornì evidenza diretta di uno degli strani fenomeni osservabili nel mondo microscopico.
L'apparato sperimentale era costituito da un forno in cui veniva fatto evaporare argento, facendo uscire un fascio di particelle da un foro praticato sulla superficie.
Il fascio veniva fatto passare attraverso un magnete con il campo orientato lungo l'asse $z$, e veniva poi analizzata la distribuzione del momento angolare di spin lungo l'asse $z$, $S_z$.
Secondo le predizioni della fisica classica, dovremmo osservare una distribuzione continua di valori per $S_z$, ma le misurazioni
rilevarono invece che i valori assunti erano soltanto due, $+ \hbar / 2$ e $- \hbar / 2$.
Essendo $S_z$ una grandezza fisica che assume valori discreti, possiamo modellare una particella di spin $\frac{1}{2}$ secondo il formalismo descritto fino a questo punto. Indicheremo gli stati

	\[
		\ket{S_z = + \hbar / 2} \quad \quad \ket{S_z = - \hbar / 2}
	\]

con, rispettivamente:

	\[
		\ket{+z} \quad \quad \ket{-z}
	\]

Possiamo quindi, ad esempio, scrivere uno stato generico $\ket{\psi}$ di una particella come:

	\begin{equation}
		\ket{\psi} =  \ket{+z} \braket{+z}{\psi} + \ket{-z} \braket{-z}{\psi}
	\end{equation}

In forma vettoriale, con la scelta ovvia della base $B_z = \{ \ket{+z}, \ket{-z} \}$:

	\begin{equation}
		\ket{\psi} =
			\left ( \begin{array}{c}
				\braket{+z}{\psi} \\
				\braket{-z}{\psi}
			\end{array} \right )_{B_z}
	\end{equation}

Naturalmente la scelta della base non \`e unica. Ad esempio, un osservatore che misurasse lo spin $S_x$ lungo l'asse $x$ dovrebbe ottenere risultati paragonabili. Se scegliamo $B_x = \{ \ket{+x}, \ket{-x} \}$ dobbiamo avere analogamente:

	\begin{equation}
		\ket{\psi} =
			\left ( \begin{array}{c}
				\braket{+x}{\psi} \\
				\braket{-x}{\psi}
			\end{array} \right )_{B_x}
	\end{equation}

\subsection{Operatori di rotazione}

Supponiamo di considerare una particella che possieda spin $S_z = + \hbar / 2$. Possiamo ruotarla sul piano $xz$ in maniera tale che al termine dell'operazione lo spin sia orientato sull'asse $x$, ovvero la particella sia diventata tale da avere $S_x = + \hbar / 2 $. Consideriamo dunque un \textit{operatore di rotazione} $\hat{R}(\frac{\pi}{2} j)$, che agisce sul \textit{ket} $\ket{+z}$ in maniera tale che:

	\begin{equation} \label{eq:rotationOperatorGeneric}
		\ket{+x} = \hat{R}(\tfrac{\pi}{2} j) \ket{+z}
	\end{equation} 

Per la \eqref{eq:proofHardAlt} abbiamo anche:

	\begin{equation} \label{eq:rotationOperatorAdjoint}
		\bra{+x} = \bra{+z} \hat{R}^\dagger(\tfrac{\pi}{2} j)
	\end{equation}

	Notiamo che $ \hat{R}( \tfrac{\pi}{2} j) $ non \`e autoaggiunto, infatti se fosse $ \hat{R}(\tfrac{\pi}{2} j) = \hat{R}^\dagger(\tfrac{\pi}{2} j)$ dovremmo avere:
	
	\begin{equation}
	 	1 = \braket{+x}{+x} = \left( \bra{+z} \hat{R}(\tfrac{\pi}{2} j) \right ) \left (\hat{R}(\tfrac{\pi}{2} j) \right ) = \mel{+z}{\hat{R}(\tfrac{\pi}{2} j)\hat{R}(\tfrac{\pi}{2} j) }{+z} 
	\end{equation} 

Ma se $ \hat{R}(\tfrac{\pi}{2} j )$ effettua una rotazione di 90 gradi attorno all'asse $y$, allora applicato due volte effettuer\`a una rotazione di 180 gradi. Allora:

	\begin{equation} \label{eq:rotationNotHermitian}
		1 = \mel{+z}{\hat{R}(\pi j ) }{+z}
	\end{equation}

Dal momento che stiamo parlando di rotazioni sul piano $xy$,

	\begin{equation}
		\hat{R}(\pi j ) \ket{+z} = \ket{-z}
	\end{equation}

che, sostituita nella \eqref{eq:rotationNotHermitian} ci porta a:

	\begin{equation}
		1 = \braket{+z}{-z} = 0
	\end{equation}

che \`e assurdo.

Applicando la \eqref{eq:rotationOperatorGeneric} e la \eqref{eq:rotationOperatorAdjoint}, possiamo scrivere:

	\begin{equation}
		1 = \braket{+x}{+x} = \left ( \bra{+z} \hat{R}^\dagger(\tfrac{\pi}{2} j) \right ) \left ( \hat{R}(\tfrac{\pi}{2} j) \ket{+z} \right ) = \braket{+z}{+z} = 1
	\end{equation}

Da cui deduciamo che l'operatore di rotazione \`e l'inverso del suo aggiunto. In altre parole, se $\hat{R}(\tfrac{\pi}{2} j)$ esegue una rotazione intorno all'asse $y$ di 90 gradi in senso orario, $\hat{R}^\dagger(\tfrac{\pi}{2} j) $ compie la medesima rotazione in senso antiorario. \\

\subsection{Generatore di rotazioni}

Consideriamo ora rotazioni attorno all'asse $z$. Supponiamo inoltre di effettuare una rotazione di un angolo infinitesimo $d \vartheta$. Una maniera pi\`u utile di esprimere l'operatore di rotazione infinitesimale \`e la seguente:

	\begin{equation} \label{eq:introducingGeneratorRot}
		\hat{R} ( { d \vartheta k } ) = 1 - \frac{i}{\hbar} \hat{J}_z d \vartheta
	\end{equation}

Dove abbiamo introdotto un nuovo operatore $\hat{J}_z$ a cui richiediamo ovviamente di essere tale da rendere RHS nella \eqref{eq:introducingGeneratorRot} tendente a $1$ per $d \vartheta$ che tende a $0$. Dimostreremo ora che l'operatore $\hat{J}_z$ \`e autoaggiunto. Prendendo la forma aggiunta della precedente, possiamo scrivere:

	\begin{equation} \label{eq:introducingGeneratorRot}
		\hat{R}^\dagger ( { d \vartheta k } ) = 1 + \frac{i}{\hbar} \hat{J}_z^\dagger d \vartheta
	\end{equation}

dove, secondo la usuale convenzione, denotiamo $\hat{J}_z^\dagger$ l'operatore aggiunto di $\hat{J}_z$. L'operatore di rotazione deve necessariamente soddisfare:

	\begin{equation}
	 	1 = \hat{R}^\dagger(d \vartheta k) \hat{R}(d \vartheta k) = \left ( 1 + \frac{i}{\hbar} \hat{J}_z^\dagger d \vartheta \right ) \left ( 1 - \frac{i}{\hbar} \hat{J}_z d \vartheta \right )
	\end{equation}

Svolgendo:

	\begin{equation} \label{eq:proofGeneratorHermitian}
		1 + \frac{i}{\hbar} \left ( \hat{J}^\dagger_z - \hat{J}_z \right ) d \vartheta + O(d \vartheta ^2) = 1
	\end{equation}

Dal momento che $d \vartheta$ \`e infinitesimo, possiamo ignorare termini di ordine superiore al primo, la \eqref{eq:proofGeneratorHermitian} \`e dunque soddisfatta solo per $\hat{J}_z = \hat{J}^\dagger_z$. Avendo descritto una rotazione di un angolo infinitesimo $d \vartheta$, possiamo scrivere rotazioni per ogni angolo finito $\vartheta$ come somma di un numero infinito di rotazioni infinitesime usando:

	\[
		d \vartheta = \lim_{N \to +\infty} \frac{\vartheta}{N}
	\]

Possiamo quindi riscrivere l'operatore di rotazione $\hat{R} (\vartheta k )$ come:

	\begin{equation}
		\hat{R} (\vartheta k ) = \lim_{N \to +\infty} \left [ 1 - \frac{i}{\hbar} \left ( \frac {\vartheta}{N} \right ) \right ] ^ N
	\end{equation}

Espandendo RHS in serie di Taylor, possiamo stabilire infine:

	\begin{equation} \label{eq:rOperatorAsExp}
		\hat{R} (\vartheta k ) = e ^ {-i \hat{J}_z \vartheta / \hbar}
	\end{equation}

Potremmo chiederci ora quale sia l'effetto di una rotazione intorno all'asse $z$ del \textit{ket} $\ket{+z}$. Fisicamente, l'effetto di una tale rotazione \`e evidente: esattamente come un vettore fatto ruotare attorno ad uno stato parallelo a quest'ultimo, lo stato deve rimanere invariato. Come abbiamo discusso precedentemente, due stati che differiscano solamente per un fattore di fase sono considerati identici, dunque ci aspettiamo che valga la seguente relazione:

	\begin{equation}
		\hat{R} (\vartheta k) \ket{+z} = e^{i \zeta} \ket{+z}
	\end{equation}

Mostreremo che la precedente \`e valida solo se

	\begin{equation} \label{eq:proofEigenketCondition}
		\hat{J}_z \ket{+z} = c \ket{+z}
	\end{equation}

dove $c$ \`e una costante, ovvero se $\ket{+z}$ \`e un autovettore di $\hat{J}_z$. Espandendo l'esponenziale nella \eqref{eq:rOperatorAsExp} in serie di Taylor si ha:

	\begin{equation} \label{eq:rotGenTaylorSeries}
		\hat{R} (\vartheta k) \ket{+z} = \left ( 1 - \frac{i \vartheta \hat{J}_z}{\hbar} + \frac{1}{2!} \left ( - \frac{i \vartheta \hat{J}_z}{\hbar} \right )^2 + ... \right ) \ket{+z}
	\end {equation}

Supponiamo per assurdo che la \eqref{eq:proofEigenketCondition} sia falsa. Dunque, $\hat{J}_z$ deve essere diverso da $c \ket{+z}$, ovvero deve contenere un termine diverso da $\ket{+z}$, ad esempio $\ket{+x}$. Ma allora i primi  due termini della successione risulterebbero valere $\ket{+z}$ pi\`u un termine in $\ket{+x}$, che non pu\`o essere eliminato perch\`e termini che contengono potenze diverse di $\vartheta$ sono linearmente indipendenti tra loro.

Si pu\`o mostrare che l'autovalore associato ad un tale autostato corrisponde esattamente al valore $S_z$, misurato dagli esperimenti di Stern-Gerlach, in formule:
 
 	\begin{equation}
 		\hat{J}_z \ket{\pm z} = \pm \frac{\hbar}{2} \ket{\pm z}
 	\end{equation}
 
Se la precedente \`e vera, allora vale:
 
 	\begin{equation}
 		\hat{J}_z^2 \ket{+z} = \hat{J}_z \frac{\hbar}{2} \ket{+z} = \left ( \frac{\hbar}{2} \right ) ^ 2 \ket{+z}
 	\end{equation}
 
e cos\`i via. Dalla \eqref{eq:rotGenTaylorSeries} ottieniamo
 
 	\begin{equation} \label{eq:rotateSpinUp}
 		\hat{R}( \varphi k) \ket{+z} = \left [ 1 - \frac{i \varphi}{2} + \frac{1}{2!} \left ( - \frac{i \varphi}{2} \right )^2 + ... \right ] \ket{+z} = e^{-i \varphi / 2} \ket{+z}
 	\end{equation}
 
in cui lo stato iniziale risulta moltiplicato per un fattore di fase globale, esattamente quello che ci aspetteremmo dal momento che lo stato non deve cambiare. Il valore della fase \`e determinato dall'autovalore corrispondente a $\ket{+z}$. Per mostrare che deve essere $\hbar / 2$, consideriamo la rotazione dello stato a spin down $\ket{-z}$ intorno all'asse $z$, ovvero $\hat{R}(\varphi k) \ket{-z}$. Con un ragionamento analogo a quello che ci ha portato a concludere che $\ket{+z}$ \`e autostato di $\hat{J}_z$, possiamo affermare che anche $\ket{-z}$ lo deve essere. Possiamo inoltre assumere che i due autovalori debbano essere diversi. Se cos\`i non fosse, applicando l'operatore di rotazione $\hat{R} (\varphi k )$ allo stato:

	\begin{equation}
		\ket{+x} = \frac{1}{\sqrt{2}} \ket{+z} + \frac{1}{\sqrt{2}} \ket{-z}
	\end{equation}

non lo ruoterebbe, dal momento che $\ket{+z}$ e $\ket{-z}$ acquisirebbero lo stesso fattore di fase, e dunque anche $\ket{+x}$ sarebbe moltiplicato per il medesimo fattore, rimanendo quindi invariato. Ma ruotando $\ket{+x}$ di un angolo $\varphi$ attorno all'asse $z$ ci aspetteremmo che lo stato cambiasse. Provando:

	\begin{equation}
		\hat{J}_z \ket{-z} = - \frac{\hbar}{2} \ket{-z}	
	\end{equation}

La \eqref{eq:rotGenTaylorSeries} ci d\`a:

	\begin{equation} \label{eq:rotateSpinDown}
		\hat{R}(\varphi k) \ket{-z} = \left [ 1 + \frac{i \varphi}{2} + \frac{1}{2!} \left( \frac{i \varphi}{2} \right )^2 + ... \right ] \ket{-z} = e^{i \varphi / 2} \ket {-z}
	\end{equation}

Da cui, per la \eqref{eq:rotateSpinUp} e la \eqref{eq:rotateSpinDown}:

	\begin{equation}
		\hat{R}(\varphi k) \ket{+x} = \frac{e^{-i \varphi / 2}}{\sqrt{2}} \ket{+z} + \frac{e^{i \varphi / 2}}{\sqrt{2}} \ket{-z}
	\end{equation}

Raccogliendo:

	\begin{equation}
		e^{-i \varphi / 2} \left ( \frac{1}{\sqrt{2}} \ket{+z} + \frac{e^{i \varphi}}{\sqrt{2}} \ket{-z} \right )
	\end{equation}

che \`e uno stato chiaramente diverso rispetto al precedente per $\varphi \neq 0$. In particolare, se poniamo $\varphi = \pi / 2$, otteniamo

	\begin{equation}
		\hat{R}(\tfrac{\pi}{2} k) = e^{-i \pi / 4} \left ( \frac{1}{\sqrt{2}} \ket{+z} + \frac{e^{i \pi / 2}}{\sqrt{2}} \ket{-z} \right )	
	\end{equation}

Ovvero:

	\begin{equation}
		e^{-i \pi / 4} \left( \frac{1}{\sqrt{2}} \ket{+z} + \frac{i}{\sqrt{2}} \ket{-z} \right ) = e^{-i \pi / 4} \ket{+y}
	\end{equation}

Dal momento che due stati che differiscono solamente per una fase globale sono identici, vediamo che ruotando lo stato $\ket{+z}$ di un angolo retto in senso antiorario attorno all'asse $z$ genera lo stato $\ket{+y}$.

La conclusione \`e sorprendente: il generatore di rotazioni intorno all'asse $z$, se applicato agli stati $\ket{+z}$ e $\ket{-z}$, risulta essere una costante per lo stato stesso; gli autovalori per i due autostati sono i valori della componente $z$ del momento angolare intrinseco che caratterizza quegli stati.

\subsection{Rappresentazione matriciale di $\hat{J}_z$}

Evidentemente, anche l'operatore $\hat{J_z}$ pu\`o essere espresso in forma matriciale. Grazie alle formule precedenti possiamo facilmente valutare:

	\begin{equation}
		\hat{J}_z = 
			\left( \begin{array}{c c}
				\mel{+z}{\hat{J}_z}{+z} & \mel{+z}{\hat{J}_z}{-z} \\
				\mel{-z}{\hat{J}_z}{+z} & \mel{-z}{\hat{J}_z}{-z} \\
			\end{array}
		\right )
	\end{equation}

Da cui:	

	\begin{equation}
		\hat{J}_z = 
			\left( \begin{array}{c c}
				(\hbar / 2) \braket{+z}{+z} & (- \hbar / 2) \braket{+z}{-z} \\
				(\hbar / 2) \braket{+z}{-z} & (- \hbar / 2) \braket{-z}{-z} 
			\end{array}
		\right )
	\end{equation}

Ovvero:
	
	\begin{equation}
		\hat{J}_z = 
			\left( \begin{array}{c c}
				\hbar / 2 & 0 \\
				0 &  - \hbar / 2 \\ 
			\end{array}
		\right )
	\end{equation}

La matrice \`e diagonale, ed ovviamente gli autovalori sono gli elementi della diagonale perch\`e stiamo usando una base di autovettori ortogonali. A questo punto, le relazioni $\hat{J}_z \ket{+z} = (\hbar / 2) \ket{+z}$ e $\hat{J}_z \ket{-z} = (- \hbar / 2) \ket{-z}$ possono essere espresse semplicemente come:

	\begin{equation}
		\hat{J}_z = 
		\left( \begin{array}{c c}
				\hbar / 2 & 0 \\
				0 &  - \hbar / 2 \\ 
			\end{array}
		\right )
		\left( \begin{array}{c}
				1 \\
				0 \\
			\end{array}
		\right ) = \frac{\hbar}{2} \left( \begin{array}{c}
				1 \\
				0 \\
			\end{array}
		\right )
	\end{equation}

e

	\begin{equation}
		\hat{J}_z = 
		\left( \begin{array}{c c}
				\hbar / 2 & 0 \\
				0 &  - \hbar / 2 \\ 
			\end{array}
		\right )
		\left( \begin{array}{c}
				0 \\
				1 \\
			\end{array}
		\right ) -\frac{\hbar}{2} \left( \begin{array}{c}
				0 \\
				1 \\
			\end{array}
		\right ) 
	\end{equation}

Possiamo inoltre scrivere:

	\begin{equation}
		\hat{J}_z = 
		\left( \begin{array}{c c}
				\hbar / 2 & 0 \\
				0 &  - \hbar / 2 \\ 
			\end{array}
		\right ) =
		\frac{\hbar}{2}
		\left( \begin{array}{c c}
				1 & 0 \\
				0 & 0 \\ 
			\end{array}
		\right ) -
		\frac{\hbar}{2}
		\left( \begin{array}{c c}
				0 & 0 \\
				0 & 1 \\ 
			\end{array}
		\right )
	\end{equation}

Da cui, scrivendo gli operatori in forma esplicita:

	\begin{equation}
		\hat{J}_z = \frac{\hbar}{2} \hat{P}_{\ket{+z}} - \frac{\hbar}{2} \hat{P}_{\ket{+z}} = \frac{\hbar}{2} \ketbra{+z}{+z} - \frac{\hbar}{2} \ketbra{-z}{-z}
	\end{equation}

\subsection{Cambiamenti di base}

	Consideriamo l'aggiunto dell'operatore di rotazione, $\hat{R}^\dagger(\vartheta)$. Supponiamo di applicare tale operatore ad un ket qualsiasi $\ket{\psi}$ per causare una sua trasformazione:

	\[
		\ket{\psi'} = \hat{R}^\dagger(\vartheta)
	\]

Come abbiamo visto, deve valere:

	\begin{equation} \label{eq:rotationAdjoint}
		\hat{R}^\dagger(\vartheta) \hat{R}(\vartheta) = 1
	\end{equation}

Ovvero:

	\begin{equation}
		\hat{R}^\dagger(\vartheta) = \hat{R} ( -\vartheta )
	\end{equation}

Possiamo scrivere $\ket{\psi'}$ nella tradizionale forma vettoriale, scegliendo la base $B_z = \{ \ket{+z}, \ket{-z} \} $:

	\begin{equation}
		\ket{\psi'} =
			\left ( \begin{array}{c}
				\mel{+z}{\hat{R}^\dagger(\vartheta)}{\psi} \\
				\mel{-z}{\hat{R}^\dagger(\vartheta)}{\psi}
			\end{array} \right )_{B_z}
	\end{equation}

Ma invece di interpretare l'operatore $\hat{R}^\dagger$ come applicato al \textit{ket} alla sua destra, lo considereremo agente sul \textit{bra} alla sua sinistra. Dal momento che:

	\begin{equation}
		\ket{\pm x} = \hat{R} ( \tfrac{\pi}{2} j ) \ket{\pm z}
	\end{equation}

Se $\hat{R}^\dagger(\vartheta)$ \`e l'aggiunto di $\hat{R} ( \vartheta )$ allora:
	
	\begin{equation}
		\bra{\pm x} = \bra{\pm z} \hat{R}^\dagger ( \tfrac{\pi}{2} j )
	\end{equation}

Quindi possiamo scrivere:

	\begin{equation} \label{eq:rotChangeBasis}
		\ket{\psi'} =
			\left ( \begin{array}{c}
				\braket{+z}{\psi'} \\
				\braket{-z}{\psi'} \\
			\end{array} \right )_{B_z} =
			\left ( \begin{array}{c}
				\mel{+z}{\hat{R}^\dagger(\vartheta)}{\psi} \\
				\mel{-z}{\hat{R}^\dagger(\vartheta)}{\psi}
			\end{array} \right )_{B_z} =
			\left ( \begin{array}{c}
				\braket{+x}{\psi'} \\
				\braket{-x}{\psi'} \\
			\end{array} \right )_{B_z}
	\end{equation}

Possiamo notare che l'ultimo termine a destra della \eqref{eq:rotChangeBasis} \`e esattamente $\ket{\psi}$ scritto in base $B_x = \{ \ket{+x}, \ket{-x} \}$. Supponiamo ora di avere un vettore $\ket{\psi}$ scritto in base $B_x$. Inserendo l'operatore identitario scritto in base $B_z$ otteniamo:

	\begin{equation}
		\left ( \begin{array}{c}
			\braket{+x}{\psi} \\
			\braket{-x}{\psi} \\
		\end{array} \right )_{B_z} = 
		\left ( \begin{array}{c c}
			\braket{+x}{+z} & \braket{+x}{-z} \\
			\braket{-x}{+z} & \braket{-x}{-z}
		\end{array} \right )
		\left ( \begin{array}{c}
			\braket{+z}{\psi} \\
			\braket{-z}{\psi} \\
		\end{array} \right )_{B_z}  
	\end{equation}

Svolgendo i conti si ha:

	\begin{equation}	
		=
		\left ( \begin{array}{c c}
			\mel{+z}{\hat{R}^\dagger ( \tfrac{\pi}{2} j )}{-z} & \mel{+z}{\hat{R}^\dagger ( \tfrac{\pi}{2}j )}{-z} \\ 
			\mel{-z}{\hat{R}^\dagger ( \tfrac{\pi}{2} j )}{-z} & \mel{-z}{\hat{R}^\dagger ( \tfrac{\pi}{2}j )}{-z} \\ 
		\end{array} \right )
		\left ( \begin{array}{c}
			\braket{+z}{\psi} \\
			\braket{-z}{\psi} \\
		\end{array} \right )_{B_z} 
	\end{equation}

Chiameremo $S^\dagger ( \tfrac{\pi}{2} j )$ la matrice:

	\[
		\left ( \begin{array}{c c}
			\mel{+z}{\hat{R}^\dagger ( \tfrac{\pi}{2} j)}{-z} & \mel{+z}{\hat{R}^\dagger ( \tfrac{\pi}{2} j)}{-z} \\ 
			\mel{-z}{\hat{R}^\dagger ( \tfrac{\pi}{2} j)}{-z} & \mel{-z}{\hat{R}^\dagger ( \tfrac{\pi}{2} j)}{-z} \\ 
		\end{array} \right )
	\]

Possiamo facilmente verificare che $S^\dagger ( \tfrac{\pi}{2} j )$ \`e la rappresentazione in base $B_z$ dell'operatore $\hat{R}^\dagger ( \tfrac{\pi}{2} j)$.

Possiamo in maniera completamente analoga svolgere la trasformazione inversa, partendo da un vettore scritto in base $B_z$ ed andando ad inserire l'operatore identit\`a in base $B_x$. Questa volta la matrice che caratterizza la trasformazione sar\`a:

	\[
		\left ( \begin{array}{c c}
			\mel{+z}{\hat{R} ( \tfrac{\pi}{2} j)}{-z} & \mel{+z}{\hat{R} ( \tfrac{\pi}{2} j)}{-z} \\ 
			\mel{-z}{\hat{R} ( \tfrac{\pi}{2} j)}{-z} & \mel{-z}{\hat{R} ( \tfrac{\pi}{2} j)}{-z} \\ 
		\end{array} \right )
	\]

La chiameremo $S( \tfrac{\pi}{2} j )$. Possiamo facilmente verificare che $S ( \tfrac{\pi}{2} j )$ rappresenta, sempre in base $B_z$, l'operatore $\hat{R} ( \tfrac{\pi}{2} j)$. Per la \eqref{eq:rotationAdjoint} si deve avere:

	\begin{equation}
	  S ( \tfrac{\pi}{2} j ) S^\dagger ( \tfrac{\pi}{2} j ) = I
	\end{equation}

Consideriamo ora un vettore generico $\hat{A}$. La sua rappresentazione in base $B_x$ \`e:

	\[
		\hat{A} = 
			\left ( \begin{array}{c c}
				\mel{+x}{\hat{A}}{+x} & \mel{+x}{A}{-x} \\
				\mel{-x}{\hat{A}}{+x} & \mel{-x}{A}{-x}
			\end{array} \right )
	\]

La matrice precedente si pu\`o riscrivere come:

	\begin{equation}
		\hat{A} = 
			\left ( \begin{array}{c c}
				\mel{+z}{ \hat{R} ^\dagger( \tfrac{\pi}{2} j) \hat{A} \hat{R} ( \tfrac{\pi}{2} j) }{+z} & 
				\mel{+z}{ \hat{R} ^\dagger( \tfrac{\pi}{2} j) \hat{A} \hat{R} ( \tfrac{\pi}{2} j) }{-z} \\
				\mel{-z}{ \hat{R} ^\dagger( \tfrac{\pi}{2} j) \hat{A} \hat{R} ( \tfrac{\pi}{2} j) }{+z} & 
				\mel{-z}{ \hat{R} ^\dagger( \tfrac{\pi}{2} j) \hat{A} \hat{R} ( \tfrac{\pi}{2} j) }{-z}
			\end{array} \right )
	\end{equation}

Prendiamo ora

	\begin{equation}
		\hat{A} = \hat{A} I = \hat{A} 
			\left ( \begin{array}{c c}
				\braket{+z}{+z} & \braket{+z}{-z} \\
				\braket{-z}{+z} & \braket{-z}{-z}
			\end{array} \right )
	\end{equation}

Consideriamo per semplicit\`a l'elemento $\hat{A}_{11}$, gli altri saranno completamente analoghi. Si ha:

	\begin{equation}
		\begin{array}{r c l}
		\hat{A}_{11} & = & \mel{+z}{\hat{R}^\dagger ( \tfrac{\pi}{2} j )}{+z} \hat{A} \mel{+z}{\hat{R} ( \tfrac{\pi}{2} j )}{+z} \\
		& + & \braket{+z}{-z} \hat{R}^\dagger ( \tfrac{\pi}{2} j ) \hat{A} \hat{R} ( \tfrac{\pi}{2} j ) \braket{+z}{+z} \\
		\end{array}
	\end{equation}

Il secondo addendo \`e zero, possiamo allora scrivere:

	\begin{equation}
		\hat{A}_{B_x} =  S^\dagger ( \tfrac{\pi}{2} j ) \hat{A}_{B_z}  S ( \tfrac{\pi}{2} j ) 
	\end{equation}

Fisicamente, abbiamo eseguito una \textit{trasfomazione passiva}: invece di ruotare uno stato tenendo fissi gli assi del sistema di riferimento, lo stato \`e rimasto fisso e gli stati sono stati ruotati.
