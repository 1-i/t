\section{Particelle di spin $\frac{1}{2}$}

\subsection{Esperimenti di Stern-Gerlach}

Avendo stabilito un formalismo matematico che ci consenta di operare sui fenomeni fisici, passiamo a considerare alcuni casi semplici.
Nel 1922, Otto Stern e Walther Gerlach effettuarono un esperimento che fornì evidenza diretta di uno degli strani fenomeni osservabili nel mondo microscopico.
L'apparato sperimentale era costituito da un forno in cui veniva fatto evaporare argento, facendo uscire un fascio di particelle da un foro praticato sulla superficie.
Il fascio veniva fatto passare attraverso un magnete con il campo orientato lungo l'asse $z$, e veniva poi analizzata la distribuzione del momento angolare di spin lungo l'asse $z$, $S_z$.
Secondo le predizioni della fisica classica, dovremmo osservare una distribuzione continua di valori per $S_z$, ma le misurazioni
rilevarono invece che i valori assunti erano soltanto due, $+ \hbar / 2$ e $- \hbar / 2$.
Essendo $S_z$ una grandezza fisica che assume valori discreti, possiamo modellare una particella di spin $\frac{1}{2}$ secondo il formalismo descritto fino a questo punto. Indicheremo gli stati

	\[
		\ket{S_z = + \hbar / 2} \quad \quad \ket{S_z = - \hbar / 2}
	\]

con, rispettivamente:

	\[
		\ket{+z} \quad \quad \ket{-z}
	\]

Possiamo quindi, ad esempio, scrivere uno stato generico $\ket{\psi}$ di una particella come:

	\begin{equation}
		\ket{\psi} =  \ket{+z} \braket{+z}{\psi} + \ket{-z} \braket{-z}{\psi}
	\end{equation}

(...)

\subsection{Operatori di rotazione}

Supponiamo di considerare una particella che possieda spin $S_z = + \hbar / 2$. Possiamo ruotarla sul piano $xz$ in maniera tale che al termine dell'operazione lo spin sia orientato sull'asse $x$, ovvero la particella sia diventata tale da avere $S_x = + \hbar / 2 $. Consideriamo dunque un \textit{operatore di rotazione} $\hat{R}(\frac{\pi}{2} j)$, che agisce sul \textit{ket} $\ket{+z}$ in maniera tale che:

	\begin{equation}
		\ket{+x} = \hat{R}(\tfrac{\pi}{2} j) \ket{+z}
	\end{equation} 

Per la \eqref{eq:proofHardAlt} abbiamo anche:

	\begin{equation}
		\bra{+x} = (\tfrac{\pi}{2} j) \bra{+z} \hat{R}^\dagger(\tfrac{\pi}{2} j)
	\end{equation}

	Notiamo che $ \hat{R}^(\tfrac{\pi}{2} j) $ non \`e autoaggiunto, infatti se fosse $ \hat{R}^(\tfrac{\pi}{2} j = \hat{R}^\dagger(\tfrac{\pi}{2}$ dovremmo avere:
	
	\begin{equation}
	 	1 = \braket{+x}{+x} = \left( \bra{+z} \hat{R}(\tfrac{\pi}{2} j) \right ) \left (\hat{R}(\tfrac{\pi}{2} j) \right ) = \mel{+z}{\hat{R}(\tfrac{\pi}{2} j)\hat{R}(\tfrac{\pi}{2} j) }{+z} 
	\end{equation} 

Ma se $ \hat{R}(\tfrac{\pi}{2} j )$ effettua una rotazione di 90 gradi attorno all'asse $y$, allora applicato due volte effettuer\`a una rotazione di 180 gradi. Allora:

	\begin{equation} \label{eq:rotationNotHermitian}
		1 = \mel{+z}{\hat{R}(\tfrac{\pi} j ) }{+z}
	\end{equation}

Dal momento che stiamo parlando di rotazioni sul piano $xy$,

	\begin{equation}
		\hat{R}(\tfrac{\pi} j ) \ket{+z} = \ket{-z}
	\end{equation}

che, sostituita nella \eqref{eq:rotationNotHermitian} ci porta a:

	\begin{equation}
		1 = \braket{+z}{-z} = 0
	\end{equation}

che \`e assurdo.
