\section{Particelle di spin $\frac{1}{2}$}

\subsection{Esperimenti di Stern-Gerlach}

Avendo stabilito un formalismo matematico che ci consenta di operare sui fenomeni fisici, passiamo a considerare alcuni casi semplici.
Nel 1922, Otto Stern e Walther Gerlach effettuarono un esperimento che fornì evidenza diretta di uno degli strani fenomeni osservabili nel mondo microscopico.
L'apparato sperimentale era costituito da un forno in cui veniva fatto evaporare argento, facendo uscire un fascio di particelle da un foro praticato sulla superficie.
Il fascio veniva fatto passare attraverso un magnete con il campo orientato lungo l'asse $z$, e veniva poi analizzata la distribuzione del momento angolare di spin lungo l'asse $z$, $S_z$.
Secondo le predizioni della fisica classica, dovremmo osservare una distribuzione continua di valori per $S_z$, ma le misurazioni
rilevarono invece che i valori assunti erano soltanto due, $+ \hbar / 2$ e $- \hbar / 2$.
Essendo $S_z$ una grandezza fisica che assume valori discreti, possiamo modellare una particella di spin $\frac{1}{2}$ secondo il formalismo descritto fino a questo punto. Indicheremo gli stati

	\[
		\ket{S_z = + \hbar / 2} \quad \quad \ket{S_z = - \hbar / 2}
	\]

con, rispettivamente:

	\[
		\ket{+z} \quad \quad \ket{-z}
	\]

Possiamo quindi, ad esempio, scrivere uno stato generico $\ket{\psi}$ di una particella come:

	\begin{equation}
		\ket{\psi} =  \ket{+z} \braket{+z}{\psi} + \ket{-z} \braket{-z}{\psi}
	\end{equation}

(...)

\subsection{Operatori di rotazione}

Supponiamo di considerare una particella che possieda spin $S_z = + \hbar / 2$. Possiamo ruotarla sul piano $xz$ in maniera tale che al termine dell'operazione lo spin sia orientato sull'asse $x$, ovvero la particella sia diventata tale da avere $S_x = + \hbar / 2 $. Consideriamo dunque un \textit{operatore di rotazione} $\hat{R}(\frac{\pi}{2} j)$, che agisce sul \textit{ket} $\ket{+z}$ in maniera tale che:

	\begin{equation} \label{eq:rotationOperatorGeneric}
		\ket{+x} = \hat{R}(\tfrac{\pi}{2} j) \ket{+z}
	\end{equation} 

Per la \eqref{eq:proofHardAlt} abbiamo anche:

	\begin{equation} \label{eq:rotationOperatorAdjoint}
		\bra{+x} = \bra{+z} \hat{R}^\dagger(\tfrac{\pi}{2} j)
	\end{equation}

	Notiamo che $ \hat{R}( \tfrac{\pi}{2} j) $ non \`e autoaggiunto, infatti se fosse $ \hat{R}(\tfrac{\pi}{2} j) = \hat{R}^\dagger(\tfrac{\pi}{2} j)$ dovremmo avere:
	
	\begin{equation}
	 	1 = \braket{+x}{+x} = \left( \bra{+z} \hat{R}(\tfrac{\pi}{2} j) \right ) \left (\hat{R}(\tfrac{\pi}{2} j) \right ) = \mel{+z}{\hat{R}(\tfrac{\pi}{2} j)\hat{R}(\tfrac{\pi}{2} j) }{+z} 
	\end{equation} 

Ma se $ \hat{R}(\tfrac{\pi}{2} j )$ effettua una rotazione di 90 gradi attorno all'asse $y$, allora applicato due volte effettuer\`a una rotazione di 180 gradi. Allora:

	\begin{equation} \label{eq:rotationNotHermitian}
		1 = \mel{+z}{\hat{R}(\pi j ) }{+z}
	\end{equation}

Dal momento che stiamo parlando di rotazioni sul piano $xy$,

	\begin{equation}
		\hat{R}(\pi j ) \ket{+z} = \ket{-z}
	\end{equation}

che, sostituita nella \eqref{eq:rotationNotHermitian} ci porta a:

	\begin{equation}
		1 = \braket{+z}{-z} = 0
	\end{equation}

che \`e assurdo.

Applicando la \eqref{eq:rotationOperatorGeneric} e la \eqref{eq:rotationOperatorAdjoint}, possiamo scrivere:

	\begin{equation}
		1 = \braket{+x}{+x} = \left ( \bra{+z} \hat{R}^\dagger(\tfrac{\pi}{2} j) \right ) \left ( \hat{R}(\tfrac{\pi}{2} j) \ket{+z} \right ) = \braket{+z}{+z} = 1
	\end{equation}

Da cui deduciamo che l'operatore di rotazione \`e l'inverso del suo aggiunto. In altre parole, se $\hat{R}(\tfrac{\pi}{2} j)$ esegue una rotazione intorno all'asse $y$ di 90 gradi in senso orario, $\hat{R}^\dagger(\tfrac{\pi}{2} j) $ compie la medesima rotazione in senso antiorario. \\

\subsection{Generatore di rotazioni}

Consideriamo ora rotazioni attorno all'asse $z$. Supponiamo inoltre di effettuare una rotazione di un angolo infinitesimo $d \vartheta$. Una maniera pi\`u utile di esprimere l'operatore di rotazione infinitesimale \`e la seguente:

	\begin{equation} \label{eq:introducingGeneratorRot}
		\hat{R} ( { d \vartheta k } ) = 1 - \frac{i}{\hbar} \hat{J}_z d \vartheta
	\end{equation}

Dove abbiamo introdotto un nuovo operatore $\hat{J}_z$ a cui richiediamo ovviamente di essere tale da rendere RHS nella \eqref{eq:introducingGeneratorRot} tendente a $1$ per $d \vartheta$ che tende a $0$. Dimostreremo ora che l'operatore $\hat{J}_z$ \`e autoaggiunto. Prendendo la forma aggiunta della precedente, possiamo scrivere:

	\begin{equation} \label{eq:introducingGeneratorRot}
		\hat{R}^\dagger ( { d \vartheta k } ) = 1 + \frac{i}{\hbar} \hat{J}_z^\dagger d \vartheta
	\end{equation}

dove, secondo la usuale convenzione, denotiamo $\hat{J}_z^\dagger$ l'operatore aggiunto di $\hat{J}_z$. L'operatore di rotazione deve necessariamente soddisfare:

	\begin{equation}
	 	1 = \hat{R}^\dagger(d \vartheta k) \hat{R}(d \vartheta k) = \left ( 1 + \frac{i}{\hbar} \hat{J}_z^\dagger d \vartheta \right ) \left ( 1 - \frac{i}{\hbar} \hat{J}_z d \vartheta \right )
	\end{equation}

Svolgendo:

	\begin{equation} \label{eq:proofGeneratorHermitian}
		1 + \frac{i}{\hbar} \left ( \hat{J}^\dagger_z - \hat{J}_z \right ) d \vartheta + O(d \vartheta ^2) = 1
	\end{equation}

Dal momento che $d \vartheta$ \`e infinitesimo, possiamo ignorare termini di ordine superiore al primo, la \eqref{eq:proofGeneratorHermitian} \`e dunque soddisfatta solo per $\hat{J}_z = \hat{J}^\dagger_z$. Avendo descritto una rotazione di un angolo infinitesimo $d \vartheta$, possiamo scrivere rotazioni per ogni angolo finito $\vartheta$ come somma di un numero infinito di rotazioni infinitesime usando:

	\[
		d \vartheta = \lim_{N \to +\infty} \frac{\vartheta}{N}
	\]

Possiamo quindi riscrivere l'operatore di rotazione $\hat{R} (\vartheta k )$ come:

	\begin{equation}
		\hat{R} (\vartheta k ) = \lim_{N \to +\infty} \left [ 1 - \frac{i}{\hbar} \left ( \frac {\vartheta}{N} \right ) \right ] ^ N
	\end{equation}

Espandendo RHS in serie di Taylor, possiamo stabilire infine:

	\begin{equation} \label{eq:rOperatorAsExp}
		\hat{R} (\vartheta k ) = e ^ {-i \hat{J}_z \vartheta / \hbar}
	\end{equation}

Potremmo chiederci ora quale sia l'effetto di una rotazione intorno all'asse $z$ del \textit{ket} $\ket{+z}$. Fisicamente, l'effetto di una tale rotazione \`e evidente: esattamente come un vettore fatto ruotare attorno ad uno stato parallelo a quest'ultimo, lo stato deve rimanere invariato. Come abbiamo discusso precedentemente, due stati che differiscano solamente per un fattore di fase sono considerati identici, dunque ci aspettiamo che valga la seguente relazione:

	\begin{equation}
		\hat{R} (\vartheta k) \ket{+z} = e^{i \zeta} \ket{+z}
	\end{equation}

Mostreremo che la precedente \`e valida solo se

	\begin{equation} \label{eq:proofEigenketCondition}
		\hat{J}_z \ket{+z} = c \ket{+z}
	\end{equation}

dove $c$ \`e una costante, ovvero se $\ket{+z}$ \`e un autovettore di $\hat{J}_z$. Espandendo l'esponenziale nella \eqref{eq:rOperatorAsExp} in serie di Taylor si ha:

	\begin{equation}
		\hat{R} (\vartheta k) \ket{+z} = \left ( 1 - \frac{i \vartheta \hat{J}_z}{\hbar} + \frac{1}{2!} \left ( - \frac{i \vartheta \hat{J}_z}{\hbar} \right )^2 + ... \right ) \ket{+z}
	\end {equation}

Supponiamo per assurdo che la \eqref{eq:proofEigenketCondition} sia falsa. Dunque, $\hat{J}_z$ deve essere diverso da $c \ket{+z}$, ovvero deve contenere un termine diverso da $\ket{+z}$, ad esempio $\ket{+x}$. Ma allora i primi  due termini della successione risulterebbero valere $\ket{+z}$ pi\`u un termine in $\ket{+x}$, che non pu\`o essere eliminato perch\`e termini che contengono potenze diverse di $\vartheta$ sono linearmente indipendenti tra loro.
