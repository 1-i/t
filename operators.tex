\subsection{Operatori identità e proiezione}

Sia $\ket{\psi}$ scritto nella base $B = (g_1, g_2, ..., g_n)$. Assumiamo il corrispondente $\bra{\psi}$ scritto nella base $B^*$. Per ogni $k$ consideriamo l'operatore

	\[
		\hat{P}_k = \ket{g_k} \bra{g_k}
	\]

È facile calcolare:

	\begin{equation}
		\hat{P}_k \ket{\psi} = \ket{g_k} \bra{g_k} \sum_i \ket{g_i} \braket{g_i}{\psi}
	\end{equation}

Infatti, portando $\ket{g_k} \bra{g_k}$ nella sommatoria si ha:

	\begin{equation}
		\hat{P}_k \ket{\psi} = \sum_i \ket{g_k} \braket{g_k}{g_i} \braket{g_i}{\psi}
	\end{equation}

Ovvero:

	\begin{equation}
		\hat{P}_k \ket{\psi} = \sum_i \delta_{ki} \ket{g_k} \braket{g_i}{\psi} =  \ket{g_k} \braket{g_k}{\psi}
	\end{equation}

In maniera completamente analoga possiamo calcolare:

	\begin{equation}
		\bra{\psi} \hat{P}_k = \braket{\psi}{g_k} \bra{g_k}
	\end{equation}

Abbiamo mostrato che $\hat{P}_k$ estrae la componente di $\ket{\psi}$ lungo $\ket{g_k}$ (o la componente di $\bra{\psi}$ lungo $\bra{g_k}$ nel caso duale): lo chiameremo quindi \textit{operatore proiezione}.
Consideriamo ora la definizione di $\hat{P}_k$

	\[
		\hat{P}_k = \ket{g_k} \bra{g_k}
	\]

e sommiamola su $k$. Otteniamo:

	\begin{equation}
		\sum_i \hat{P}_k = \sum_i \ket{g_k} \bra{g_k}
	\end{equation}

Applichiamo ora ad entrambe le parti dell'uguaglianza un vettore $\ket{\psi}$:

	\begin{equation} \label{eq:identityDeriv1}
		\left ( \sum_i \hat{P}_k \right ) \ket{\psi} = \sum_i \ket{g_k} \braket{g_k}{\psi}
	\end{equation}

RHS della \eqref{eq:identityDeriv1} è, per la \eqref{eq:ketsSuperposition}, il vettore $\ket{\psi}$:

	\begin{equation} \label{eq:identityDeriv2}
		\left ( \sum_i \hat{P}_k \right ) \ket{\psi} = \ket{\psi}
	\end{equation}

Dalla \eqref{eq:identityDeriv2} deduciamo che l'operatore $\left ( \sum_i \hat{P}_k \right )$ lascia immutato il \textit{ket} a cui viene applicato. Analoghi calcoli ci portano a scrivere:

	\begin{equation}
		\bra{\psi} \left ( \sum_i \hat{P}_k \right ) = \bra{\psi}
	\end{equation}

da cui si osserva che la medesima relazione vale per i \textit{bra}. $\left ( \sum_i \hat{P}_k \right )$ è pertanto l'operatore identità, che indichamo con $\hat{1}$:

	\[
		\hat{1} =  \left ( \sum_i \hat{P}_k \right ) = \sum_i \ket{g_k} \bra{g_k}
	\]

Le rappresentazioni matriciali dei \textit{ket} e dei \textit{bra} che abbiamo dato precedentemente, ci consentono di scrivere facilmente anche gli operatori proiezione ed identità sotto forma di matrice. Scrivendo infatti la \eqref{eq:ketbraMatrixProduct} con $\ket{\psi} = \ket{g_k}$, $\bra{\varphi} = \bra{g_k}$ abbiamo:

	\begin{equation}
		\ket{g_k}\bra{g_k} = 
		\left ( \begin{array}{c c c c}
			\braket{g_1}{g_k} \braket{g_k}{g_1} & \braket{g_1}{g_k}\braket{g_k}{g_2} & ... & \braket{g_1}{g_k} \braket{g_k}{g_n} \\
			\braket{g_2}{g_k} \braket{g_k}{g_1} & \ddots &  & \\
			\vdots & & & \\
			\braket{g_n}{g_k} \braket{g_k}{g_1}& & & \braket{g_n}{g_k} \braket{g_k}{g_n}
		\end{array} \right )
	\end{equation}

ovvero la matrice che vale $0$ ovunque, $1$ sul $k$-esimo elemento della diagonale principale. Quindi,

	\begin{equation}
		\hat{P}_1 = 
		\left ( \begin{array}{c c c c}
			1 & & & \\
			& 0 & & \\
			& & \ddots & \\
			& & & 0 \\
		\end{array} \right ), \quad ..., \quad
		\hat{P}_n = 
		\left ( \begin{array}{c c c c}
			0 & & & \\
			& 0 & & \\
			& & \ddots & \\
			& & & 1 \\
		\end{array} \right )
	\end{equation}

L'operatore identità può allora essere scritto semplicemente come:

	\begin{equation}
		\hat{1} = \sum_k \hat{P}_k =
		\left ( \begin{array}{c c c c}
			1 & & & \\
			& 1 & & \\
			& & \ddots & \\
			& & & 1 \\
		\end{array} \right ) = I_n
	\end{equation}
