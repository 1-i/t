\section{Sistemi di due particelle a spin 1/2}

	Passiamo ora a considerare un sistema di due particelle, ciascuna con spin $\tfrac{1}{2}$. Sistemi di questo tipo sono estremamente comuni, ad esempio l'interazione di un protone e di un elettrone. La scelta pi\`u naturale della base di riferimento \`e quella di rappresentare gli stati con il valore di $S_z$ per ciascuna delle particelle:

	\begin{equation} \label{eq:twoPartState}
		\ket{+z, +z} \quad \ket{+z, -z} \quad \ket{-z, +z} \quad \ket{-z, -z}
	\end{equation}	

Un'altra scelta possible \`e, ad esempio, indicare lo stato della prima particella con il suo valore di $S_x$, quello della seconda con il valore di $S_z$:

	\begin{equation}
		\ket{+z, +x} \quad \ket{+z, -x} \quad \ket{-z, +x} \quad \ket{-z, -x}
	\end{equation}

Siccome entrambi gli insiemi formano delle basi dello spazio dei \textit{ket}, siamo in grado di sovrapporre gli stati della prima per ottenere quelli della seconda. Ad esempio:

	\begin{equation}
		\ket{+x, +z} = \frac{1}{\sqrt{2}}\ket{+z, +z} + \frac{1}{\sqrt{2}}\ket{-z, +z}
	\end{equation}	

Un altro modo di effettuare la trasformazione \`e quello di applicare i generatori di rotazione per ciascuna delle particelle. Indicheremo i generatori della prima particella che, come abbiamo visto, coincidono con gli operatori associati al momento angolare con:

	\begin{equation}
		\hat{S}_1 = \hat{S}_{1x} i + \hat{S}_{1y} j + \hat{S}_{1z} k
	\end{equation}

E similmente per la seconda particella. Dal momento che possiamo ruotare le due particelle indipendentemente l'una dall'altra, i due generatori devono commutare:

	\begin{equation}
		\left [ \hat{S}_1, \hat{S}_2 \right ] = 0
	\end{equation}

(...)

\subsection{Paradosso di Einstein-Podolsky-Rosen e disuguaglianze di Bell}

	Consideriamo una particella a spin-0 a riposo, e supponiamo che decada in due particelle, ciascuna a spin-$\tfrac{1}{2}$. Perch\`e la quantit\`a di moto totale sia conservata, le due particelle devono muoversi in direzioni opposte. Inoltre, perch\`e il momento angolare totale sia conservato, le due particelle devono trovarsi nello stato $\ket{0, 0}$. Supponiamo che due sperimentatori $A$ e $B$ siano posizionati sulle traiettorie delle due particelle, con lo scopo di effettuare una misura dello spin $S_z$ sulle relative particelle.
Dal momento che, come abbiamo visto, lo stato $\ket{0, 0}$ si scrive:

	\[
		\ket{0, 0}  = \frac{1}{\sqrt{2}} \ket{+z, -z} - \frac{1}{\sqrt{2}} \ket{-z, +z}
	\]

Dalla precedente possiamo vedere che se $A$ esegue una misurazione, ha il 50\% di probabilit\`a di trovare $S_{1z} = \hbar / 2$ e il 50\% di probabilit\`a di trovare $S_{1z} = - \hbar / 2$. La medesima situazione si verifica, ovviamente, se gli sperimentatori decidono di misurare lo spin lungo altri assi. Siccome lo stato globale del sistema deve essere necessariamente $\ket{0, 0}$, la misurazione di $S_z$ determina globalmente lo stato del sistema. In altre parole, conoscendo il risultato della misura di $A$, siamo in grado di prevedere con certezza quale sar\`a il valore rilevato da $B$. Il fatto sconcertante \`e che l'argomento che abbiamo mostrato fa alcuna assunzione sul modo in cui le particelle interagiscono: anche se la particella misurata da $B$ fosse in una posizione sconosciuta da $A$, una singola misura determina lo stato di \textit{entrambe}, in netta contraddizione con il \textit{principio di localit\`a} ovvero l'assunzione, completamente ragionevole nella nostra esperienza ed adottata anche nella fisica classica, che un oggetto sia influenzato soltanto dalle sue immediate vicinanze e non da altri oggetti molto distanti nello spazio. 
Quest'ultimo punto in particolare venne interpretato come completamente irragionevole dai fisici dell'epoca, si tent\`o infatti di superare il paradosso per mezzo delle cosiddette \textit{teorie delle variabili nascoste}.

Il concetto fondamentale alla base di una teoria a variabili nascoste \`e che a differenza dell'interpretazione convenzionale della meccanica quantistica, ogni particella porti con s\'e una serie di variabili che determinano univocamente il risultato della misura di qualsiasi grandezza fisica sulla particella stessa.

Prendendo ad esempio la situazione descritta dal paradosso di Einstein-Podolsky-Rosen, nella teoria che abbiamo finora esposto, ciascuna delle due particelle generate dal decadimento non ha uno spin definito: le misure di $S_z$ non sono univocamente determinate un istante prima della misura; sono al contrario fissate dal procedimento di misura stesso. Infatti, un \textit{ket} di stato non rappresenta una miscela statistica di stati, ma una predizione di natura probabilistica sugli esiti delle misure.

In una teoria a variabili nascoste, associamo a ciascuna particella il valore che otterremmo misurando, ad esempio $S_x$ o $S_z$. Dal momento che gli spin $S_x$, $S_z$ possono essere misurati indipendentemente e forniscono valori $\hbar / 2$ e $-\hbar / 2$, per ogni particella abbiamo quattro casi:

	\begin{equation}
		\{ +z, +x \} \quad \{ +z, -x \} \quad \{ -z, +x \} \quad \{ -z, -x \}
	\end{equation}

Dove $\{ \cdot, \cdot \} $ indica i valori dello spin lungo $z$ ed $x$ rispettivamente. Una particella che sia, ad esempio, di tipo $\{ +z, -x \}$ avr\`a \textit{sempre} $S_z = + \hbar / 2$ e $S_x = - \hbar / 2$. Dai risultati della meccanica quantistica possiamo immediatamente dedurre che uno sperimentatore non pu\`o misurare contemporaneamente $S_x$ ed $S_z$ sulla medesima particella dal momento che i due operatori non commutano. Per evitare un'immediata contraddizione tra le due teorie, imponiamo che la misura di uno dei due valori implichi la rinuncia a misurare l'altro: il risultato della misura \`e predeterminato da variabili appunto ``nascoste'' che possiamo determinare soltanto una alla volta.

Se consideriamo le due particelle ottenute dal decadimento di una particella a spin-0, per la conservazione del momento angolare avremo quattro possibilit\`a per i tipi delle due particelle:

	\begin{equation}
		\begin{array}{c c c}
			\{ +z, +x \} & e & \{ -z, -x \} \\
			\{ +z, -x \} & e & \{ -z, +x \} \\
			\{ -z, +x \} & e & \{ +z, -x \} \\
			\{ -z, -x \} & e & \{ +z, +x \} \\
		\end{array}
	\end{equation}

Vogliamo dimostrare che questo modello produce risultati contraddittori rispetto a quelli della meccanica quantistica. Consideriamo una variante del precedente esperimento in cui due sperimentatori $A$, $B$ eseguono misurazioni su tre assi $a, b, c$, complanari ma non ortogonali. Ogni particella apparterr\`a ad uno dei tipi

	\[
		\{ \pm a, \pm b, \pm c\}
	\]

dove la terna denota lo spin ($\pm \hbar / 2$) su ciascuno degli assi. Per la conservazione del momento angolare, il decadimento genera una particella $p_1$ di tipo $\{ \pm a, \pm b, \pm c \}$, la corrispondente particella $p_2$ dovr\`a essere di tipo $ \{ \mp a, \mp b, \mp c \}$. Supponiamo di far ripetere il decadimento di una particella a spin-0 $N$ volte. Abbiamo 8 casi per le particelle $p_1, p_2$ che supponiamo verificarsi per $N_1, N_2, ..., N_8$ volte:

	\begin{equation}
		\begin{array} {l l l}
				& p_1 & p_2 \\
			N_1 & \{ +a, +b, +c \} & \{ -a, -b, -c \} \\
			N_2 & \{ +a, +b, -c \} & \{ -a, -b, +c \} \\
			N_3 & \{ +a, -b, +c \} & \{ -a, +b, -c \} \\
			N_4 & \{ +a, -b, -c \} & \{ -a, +b, +c \} \\
			N_5 & \{ -a, +b, +c \} & \{ +a, -b, -c \} \\
			N_6 & \{ -a, +b, -c \} & \{ +a, -b, +c \} \\
			N_7 & \{ -a, -b, +c \} & \{ +a, +b, -c \} \\
			N_8 & \{ -a, -b, -c \} & \{ +a, +b, +c \} \\
		\end{array}
	\end{equation}

Supponiamo che $A$, $B$ eseguano due misurazioni su assi scelti casualmente ma diversi tra di loro. Vogliamo determinare la probabilit\`a che $A$, $B$ rilevino valori opposti.

Innanzitutto, notiamo che se ci troviamo nei casi $N_1$ o $N_8$ la probabilit\`a \`e $1$: per qualsiasi scelta della coppia di assi su cui effettuare la misurazione i risultati saranno opposti.

Per ciascuno dei casi $N_2, ..., N_7$, consideriamo la misurazione effettuata da $A$. Ci sono due assi $x_1, x_2$ su cui il valore di spin \`e $s$, ed uno ($x_3$) su cui vale $-s$ per $s \in \{ \hbar/2, -\hbar/2 \}$. Secondo le regole del nostro esperimento, se $A$ sceglie $x_3$, $B$ deve scegliere $x_1$ o $x_2$; ma siccome $A$ avrebbe misurato $s$ su ciascuno di questi, $B$ dovr\`a misurare $-s$. Ne consegue che se $A$ sceglie $x_3$, $A$ e $B$ misureranno valori uguali. Supponiamo invece che $A$ scelga uno tra $x_1$ ed $x_2$, senza perdita di generalit\`a $x_1$. $B$ pu\`o scegliere con probabilit\`a $1/2$ $x_2$ o $x_3$. Se sceglie $x_2$ misurer\`a $-s$, valore opposto a quello di $A$, se sceglie $x_3$ misurer\`a invece $s$, cio\`e lo stesso valore. Riassumendo, la probabilit\`a che $A$ e $B$ trovino valori diversi nei casi $N_2, ..., N_7$ \`e data da:

	\begin{equation}
		\Pr(N_{2 \rightarrow 7}) = \frac{1}{3} \times 0 + 2 \times \frac{1}{3} \times \frac{1}{2} = \frac{1}{3}
	\end{equation}

Utilizzando la precedente possiamo scrivere la probabilit\`a globale come:

	\begin{equation}
		\Pr(N_{1 \rightarrow 8}) = \frac{\sum_i N_i \Pr(N_i)}{\sum_i N_i}
	\end{equation}

Notiamo che:

	\begin{equation}
		\Pr(N_{1 \rightarrow 8}) = \frac{1}{N} \left ( N_1 + N_8 + \sum_{i=2}^7 \frac{1}{3} N_i \right )
	\end{equation}

da cui:

	\begin{equation}
		\Pr(N_{1 \rightarrow 8}) \ge \frac{1}{N} \left ( \frac{N_1}{3} + \frac{N_8}{3} + \sum_{i=2}^7 \frac{1}{3} N_i \right )
		= \frac{1}{N} \left ( \frac{N}{3} \right ) = \frac{1}{3}
	\end{equation}

Vediamo ora la predizione della meccanica quantistica in una situazione analoga. Rappresentiamo lo stato $\ket{0, 0}$ come:

	\begin{equation}
		\ket{0, 0} = \frac{1}{\sqrt{2}} \ket{+a, -a} - \frac{1}{\sqrt{2}} \ket{-a, +a}
	\end{equation}

L'ampiezza di probabilit\`a di trovare la particella $p_1$ con $S_{1a} = - \hbar /2$ e la particella $p_2$ con $S_{2b} = \hbar / 2$ \`e data:

	\begin{equation}
		\braket{-a, +b}{0, 0}  =  \frac{1}{\sqrt{2}} \braket{-a, +b}{+a, -a} - \frac{1}{\sqrt{2}} \braket{-a, +b}{-a, +a} \\
	\end{equation}

Il primo termine \`e zero, abbiamo quindi:

	\begin{equation}
		\braket{-a, +b}{0, 0} = - \frac{1}{\sqrt{2}} \braket{-a, +b}{-a, +a} = - \frac{1}{2} \left ( \braket{-a}{-a}_1 \right )
			\left( \braket{+b}{+a}_2 \right )
	\end{equation}

Evidentemente $\braket{-a}{-a}_1 = 1$, quindi:

	\begin{equation}
		\braket{-a, +b}{0, 0} = - \frac{1}{\sqrt{2}} \braket{+b}{+a}
	\end{equation}

Se $\ket{n}$ forma un angolo $\vartheta$ con l'asse $z$ vale:

	\begin{equation}
		\braket{+z}{+n} = \cos \frac{\vartheta}{2}
	\end{equation}

Quindi:

	\begin{equation}
		\braket{+b}{+a} = \cos \frac{\vartheta_{ab}}{2}
	\end{equation}

La probabilit\`a effettiva di trovare le particelle nello stato $\ket{-a, +b}$ \`e dunque

	\begin{equation}
		\left | \braket{+a, -b}{0, 0} \right | ^ 2  = \frac{1}{2} \cos^2 \frac{\vartheta_{ab}}{2}
	\end{equation}

In maniera completamente analoga:

	\begin{equation}
		\left | \braket{-a, +b}{0, 0} \right | ^ 2  = \frac{1}{2} \cos^2 \frac{\vartheta_{ab}}{2}
	\end{equation}

Dunque, la probabilit\`a complessiva di ottenere spin di segno opposto misurando lo spin lungo $a$ di $p_1$ e lungo $b$ di $p_2$ \`e data da:

	\begin{equation} \label{eq:paradox}
		2 \times \frac{1}{2} \cos ^2 \frac{\vartheta_{ab}}{2} = \cos^2 \frac{\vartheta_{ab}}{2}
	\end{equation}

Supponiamo di prendere gli assi $a, b, c$ ruotati di 120 gradi l'uno rispetto all'altro. Allora, la situazione \`e completamente simmetrica a quella che abbiamo descritto precedentemente per ogni possibile scelta di assi distinti. Valutando la \eqref{eq:paradox} per $\vartheta_{ab} = \pi / 3$ si ottiene:

	\begin{equation}
		cos^2 \frac{\pi}{6} = \frac{1}{4}
	\end{equation}

L'equazione precedente \`e in completo disaccordo con la predizione effettuata dalla teoria a variabili nascoste che abbiamo sviluppato fino ad ora: una delle due deve essere sbagliata ed un esperimento in laboratorio pu\'o decidere quale delle due sia in errore.
