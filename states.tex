\subsection{Stato generico e sovrapposizione di stati}

Consideriamo una grandezza fisica $G$ la cui misurazione fornisce i valori discreti $S = \{g_1, g_2, ..., g_n\}$. \`E un postulato della teoria che a $G$ sia associato un operatore $\hat{G}$ sullo spazio $\mathcal{H}$, lineare ed autoaggiunto, ovvero tale che

    \begin{equation}
        \left ( \bra{\varphi} \hat{G} \right ) \ket{\psi} =
        \bra{\varphi} \left ( \hat{G} \ket{\psi} \right ) \overset{def}{=}
        \mel{\varphi}{\hat{G}}{\psi}
    \end{equation}

Notiamo che per qualsiasi operatore autoaggiunto $\hat{S}$ deve valere:
    
    \begin{equation} \label{eq:proofHard0}
        \hat{S} \ket{\psi} = \ket{\varphi} \iff \bra{\psi} \hat{S} = \bra{\varphi}
    \end{equation}
    
Infatti:

    \begin{equation} \label{eq:proofHard1}
        \bra{\psi} \hat{S} = \bra{\varphi} \iff 
        \mel{\psi}{\hat{S}}{\psi} = \braket{\varphi}{\psi} \iff
        \mel{\psi}{\hat{S}}{\psi}^* = \braket{\psi}{\varphi}
    \end{equation}  
    
Ma interpretando $\mel{\psi}{\hat{S}}{\psi}$ come $\left ( \bra{\psi} \hat{S} \right ) \ket{    \psi}$:

    \begin{equation} \label{eq:proofHard2}
        \mel{\psi}{\hat{S}}{\psi} = \left \langle \hat{S} \psi, \psi \right \rangle =
        \left \langle \psi, \hat{S} \psi \right \rangle^* = \mel{\psi}{\hat{S}}{\psi}^*
    \end{equation}
    
A questo punto, sostituiamo la \eqref{eq:proofHard2} nella \eqref{eq:proofHard1}:

    \begin{equation}
        \bra{\psi} \hat{S} = \bra{\varphi} \iff \mel{\psi}{\hat{S}}{\psi} = \braket{\psi}{\varphi} \iff \hat{S} \ket{\psi}  = \ket{\varphi}
    \end{equation}

che \`e quello che volevamo dimostrare. Inoltre, analizzando la \eqref{eq:proofHard2}, scopriamo un altro fatto interessante, ovvero che se $\hat{S}$ \`e un operatore autoaggiunto, allora $\mel{\psi}{S}{\psi}$ \`e un numero reale, nessun numero complesso non reale può essere infatti uguale al suo coniugato.

Con un argomento analogo possiamo dimostrare in generale che se $\hat{S}$ \`e un operatore, non necessariamente autoaggiunto, allora se $\hat{S} ^ \dagger$ \`e il suo aggiunto, ovvero il vettore tale che:
    
	\begin{equation}
        	\left ( \bra{\varphi} \hat{S} \right ) \ket{\psi} =
        	\bra{\varphi} \left ( \hat{S}^\dagger \ket{\psi} \right ) 
	\end{equation}

allora:
    
    \begin{equation} \label{eq:proofHardAlt}
        \hat{S} \ket{\psi} = \ket{\varphi} \iff \bra{\psi} \hat{S}^\dagger = \bra{\varphi}
    \end{equation}

Per il teorema spettrale, deve esistere una base $B$ di $\mathcal{H}$ che sia ortonormale e composta da autovettori di $G$. Si può supporre che $G$ sia stato scelto in maniera tale che gli autovalori associati agli autovettori in $B$ siano gli elementi di $S$.
In tal caso, indichiamo gli autovettori corrispondenti a $g_1, g_2, ..., g_n$ con $\ket{g_1}, \ket{g_2}, ..., \ket{g_n}$. Siamo autorizzati a supporre che tali autovettori siano \textit{ket} perché se costituiscono una base ortonormale devono necessariamente avere norma unitaria. \\
Consistentemente con la loro natura di grandezze fisiche, i $g_i$ devono essere tutti reali. Per dimostrarlo, scriviamo l'equazione agli autovalori:

    \begin{equation} \label{eq:realEigenvaluesProof0}
        \hat{G} \ket{g_i} = g_i \ket{g_i}
    \end{equation}
    
e consideriamola applicata a $\bra{g_i}$. Otteniamo:

    \begin{equation}
        \mel{g_i}{\hat{G}}{g_i} = \braket{g_i}{g_i}g_i
    \end{equation}
    
Ma per quanto detto prima $\mel{g_i}{\hat{G}}{g_i} \in \mathbb{R}$, quindi, siccome

    \[
        \braket{g_i}{g_i} = 1
    \]
    
anche $g_1 \in \mathbb{R}$, come volevamo dimostrare.
    
Il fatto che $B = ( \ket{g_1}, \ket{g_2}, ..., \ket{g_n} ) $ sia ortonormale si può scrivere sinteticamente come: 

	\begin{equation} \label{eq:orthonormalBasis}
		\braket{g_i}{g_j} = \delta_{ij}
	\end{equation}

Dove:

 	\begin{equation}
		\delta_{ij} = \left \{ \begin{array}{l}
				0 \quad \text{se} \quad i \neq j \\
				1 \quad \text{se} \quad i = j \\
			\end{array}
		\right.
	\end{equation}


Uno stato quantico generico, espresso sotto forma di un \textit{ket} $\ket{\psi}$, può dunque essere scritto nella forma:

	\begin{equation} \label{eq:ketAsVector}
		\ket{\psi} = c_1 \ket{g_1} + c_2 \ket{g_2} + ... = \sum_i c_i \ket{g_i}
	\end{equation}

Dove i coefficienti $(c_1, c_2, ..., c_n)$ dipendono dallo stato che stiamo considerando. La \eqref{eq:ketAsVector} può essere riscritta in una maniera più suggestiva che, come vedremo, consente una singolare interpretazione dal punto di vista fisico. Prendiamo uno dei \textit{ket} $\ket{g_k}$ e consideriamo il suo duale $\bra{g_k}$. Applicando $\bra{g_k}$ ad entrambi i termini otteniamo:

	\begin{equation}
		\bra{g_k}(\ket{\psi}) = \bra{g_k}(\sum_i c_i \ket{g_i})
	\end{equation}

Per la \eqref{eq:linearityKets} abbiamo:

	\begin{equation}
		\braket{g_k}{\psi} = \sum_i c_i \braket{g_k}{g_i}
	\end{equation}

Che per la \eqref{eq:orthonormalBasis} diventa:

	\begin{equation} \label{eq:componentsAsBrakets}
		\braket{g_k}{\psi} = c_k
	\end{equation}

A questo punto, per ogni $k$, sostituiamo la \eqref{eq:componentsAsBrakets} nella \eqref{eq:ketAsVector}. Abbiamo:

	\begin{equation} \label{eq:ketsSuperposition}
		\ket{\psi} = \ket{g_1} \braket{g_1}{\psi} + \ket{g_2} \braket{g_2}{\psi} + ... = \sum_i \ket{g_i} \braket{g_i}{\psi}
	\end{equation}

Possiamo ottenere una interessante conclusione applicando $\bra{\psi}$ alla \eqref{eq:ketsSuperposition}. Si ha infatti:

	\begin{equation}
		1 = \braket{\psi}{\psi} = \sum_i \braket{\psi}{g_i} \braket{g_i}{\psi}
	\end{equation}

Ovvero, per la \eqref{eq:conjugateBrakets}:

	\begin{equation} \label{eq:probabilitySum}
		\sum_i \left | \braket{g_i}{\psi} \right |^2 = 1
	\end{equation}

Vista la relazione di dualità intercorrente tra i \textit{ket} ed i \textit{bra}, possiamo aspettarci di ottenere una relazione simile alla \eqref{eq:ketsSuperposition} che coinvolga i \textit{bra} invece dei \textit{ket}. Consideriamo un \textit{bra} generico $\bra{\varphi}$, ed applichiamolo ad entrambi i lati dell'uguaglianza. Si ha:

	\begin{equation}
		\braket{\varphi}{\psi} = \sum_i \left( \braket{\varphi}{g_i} \braket{g_i}{\psi} \right)
	\end{equation}

Per linearità possiamo riscrivere la precedente come:

	\begin{equation} \label{eq:derivation1}
		\bra{\varphi}(\ket{\psi}) = \left( \sum_i \braket{\varphi}{g_i} \bra{g_i} \right) (\ket{\psi})
	\end{equation}

Il termine in sommatoria nella \eqref{eq:derivation1} \`e un \textit{bra} moltiplicato per uno scalare, dunque esso stesso un \textit{bra}. Dal momento che la relazione espressa dalla \eqref{eq:ketsSuperposition} \`e generale e valida per un qualsiasi scelta di $\ket{\psi}$, anche la \eqref{eq:derivation1} deve valere per ogni $\ket{\psi}$. Ma questo significa che i due funzionali lineari

	\[
		\bra{\varphi}
	\]

e

	\[
		\sum_i \braket{\varphi}{g_i} \bra{g_i}
	\]


assumono il medesimo valore se applicati sullo stesso vettore, e devono pertanto essere uguali. Possiamo quindi scrivere:

	\begin{equation} \label{eq:brasSuperposition}
		\bra{\psi} = \braket{\psi}{g_1} \bra{g_1} + \braket{\psi}{g_2} \bra{g_2} + ... = \sum_i \braket{\psi}{g_i} \bra{g_i}
	\end{equation}

Confrontando la \eqref{eq:ketsSuperposition} e la \eqref{eq:brasSuperposition}, possiamo verificare che i coefficienti di $\ket{\psi}$ nella base $B = (\ket{g_1}, \ket{g_2},... \ket{g_n})$ sono i complessi coniugati dei corrispondenti coefficienti di $\bra{\psi}$ nella base duale $B^*$.

\subsection{Interpretazione fisica del formalismo}

La notazione bra-ket, oltre ad essere un utile strumento matematico, rappresenta una chiave di interpretazione del significato fisico delle predizioni della meccanica quantistica. Come già detto, un vettore rappresentato da un \textit{ket} $\ket{\psi}$ contiene il massimo dell'informazione che possiamo conoscere riguardo ad un determinato stato. 

Il numero complesso 

	\[
		\braket{\varphi}{\psi}
	\]

viene interpretato come l' \textit{ampiezza di probabilità} per una particella nello stato $\ket{\psi}$ di trovarsi nello stato $\ket{\varphi}$. L'\textit{ampiezza di probabilità} \`e collegata alla probabilità in modo analogo a come sono collegate l'ampiezza e l'intensità di un'onda: la grandezza fisica immediatamente percepibile \`e l'intensità, ma le interazioni sono descritte in termini di ampiezza.

La probabilità di trovare una particella nello stato $\bra{\varphi}$ quando viene effettuata una misurazione su una particella nello stato $\bra{\psi}$ \`e data da:

	\begin{equation} \label{eq:braKetAsProbability}
		\braket{\varphi}{\psi} \braket{\psi}{\varphi} = \left | \braket{g_i}{\psi} \right |^2
	\end{equation}

Notiamo che l'interpretazione data dalla \eqref{eq:braKetAsProbability} \`e consistente con la relazione formale espressa dalla \eqref{eq:probabilitySum}. Infatti, se una grandezza fisica $G$ può assumere i valori discreti $g_1, g_2, ..., g_n$, effettuando una misurazione di $G$ su una particella nello stato $\ket{\psi}$ possiamo non essere certi del risultato che otterremo, ma di certo dovremmo misurare uno dei $g_i$. In altre parole, detta $p_i$ la probabilità di ottenere il valore $g_i$ da tale misurazione, la somma di tutti i $p_i$ non potrà che essere 1.

Un'ulteriore coincidenza con la realtà fisica si ottiene dall'analisi della \eqref{eq:orthonormalBasis}. Difatti, una particella si trova nello stato $\ket{g_k}$ se il suo valore della grandezza $G$ \`e conosciuto essere $g_k$. Ci aspettiamo dunque che se $i \neq j$ la probabilità di ottenere il valore $g_i$ per $G$ su una particella nello stato $\ket{g_j}$ sia zero, e conseguentemente sia zero la relativa \textit{ampiezza di probabilità} $\braket{g_i}{g_j}$.

Analogamente, \`e lecito aspettarsi che l'\textit{ampiezza di probabilità} di ottenere il valore $g_i$ per $G$ su una particella nello stato $\ket{g_i}$ sia non-nulla. La scelta (convenzionale) di avere $\braket{g_i}{g_j} = 1$ \`e data dalla maggiore semplicità nel lavorare con una base ortonormale.

L'interpretazione probabilistica della \eqref{eq:braKetAsProbability} ci consente di scrivere il \textit{valore atteso} di una misurazione di $G$ su uno stato $\ket{\psi}$, che indichiamo con $\left \langle G \right \rangle$ come:

	\begin{equation}
		\left \langle G \right \rangle = \sum_i g_i \left | \braket{g_i}{\psi} \right |^2
	\end{equation}

Possiamo anche vedere che se $G$ \`e una grandezza osservabile ed $\hat{G}$ \`e l'operatore associato a quella osservabile, allora il valore atteso $\left \langle G \right \rangle$ si può scrivere come

	\[
		\mel{\psi}{\hat{A}}{\psi}
	\]

Infatti, possiamo scrivere:

	\begin{equation}
		\mel{\psi}{\hat{A}}{\psi} = \left ( \sum c_i^* \bra{g_i} \right ) \hat{A} \left ( \sum c_i \ket{g_i} \right )
	\end{equation}

Da cui, dal momento che i vari $g_i$ sono autovettori di $G$:

	\begin{equation}
		\left ( \sum c_i^* \bra{a_1} \right ) \left ( \sum c_i g_i \ket{g_i} \right )
	\end{equation}

che \`e uguale a:

	\begin{equation}
		\sum \left | c_i \right | ^2 = \left \langle G \right \rangle
	\end{equation}

Ovvero esattamente quello che volevamo determinare.

Scriveremo inoltre l'\textit{incertezza} di $G$, che indichiamo con $\Delta G$, come la deviazione standard delle misurazioni. Dunque abbiamo:

	\begin{equation}
		(\Delta G)^2 = \left \langle (G - \left \langle G \right \rangle) ^2 \right \rangle
	\end{equation}

Ovvero:

	\begin{equation}
		(\Delta G)^2 = \left \langle G^2 - 2 G \left \langle G \right \rangle + \left \langle G \right \rangle ^2 \right \rangle
	\end{equation}

Da cui:

	\begin{equation}
		(\Delta G)^2 = \left \langle G^2 \right \rangle - 2 \left \langle G \right \rangle \left \langle G \right \rangle + \left \langle G^2 \right \rangle^2
	\end{equation}

Infine:

	\begin{equation}
		(\Delta G) = \sqrt{ \left \langle G^2 \right \rangle - \left \langle G \right \rangle ^ 2 }
	\end{equation}
